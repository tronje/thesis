\documentclass[
    fontsize=12pt,
    headings=small,
    parskip=half,
    bibliography=totoc,
    numbers=noenddot,
    open=any
]{scrreprt}

\usepackage[utf8]{inputenc}
\usepackage{datetime2}
\DTMlangsetup[en-US]{abbr}

\usepackage[titletoc,title]{appendix}

\usepackage[
    autostyle,
    ]{csquotes}
\usepackage{amsmath}
\usepackage{amssymb}
\usepackage{pifont}
\usepackage[table]{xcolor}
\usepackage{marvosym}
\usepackage{graphicx}
\usepackage{parskip}
\usepackage{hyperref}
\usepackage{setspace}
\usepackage{acronym}
% \usepackage{float}
\usepackage[scaled=.92]{helvet}
\usepackage{courier}
\usepackage[
    a4paper,
    margin=2.54cm,
    marginparwidth=2.0cm,
    footskip=1.0cm
]{geometry}

\usepackage[
    url=true,
    style=numeric,
    alldates=long,
    dateabbrev=false
]{biblatex}

\addbibresource{literature.bib}

% \pagestyle{plain}
\urlstyle{rm}

\definecolor{hlinegray}{gray}{0.8}

\newcommand{\cmark}{\textcolor{green}{\ding{51}}}
% \newcommand{\xmark}{\textcolor{red}{\ding{55}}}
\newcommand{\xmark}{\textcolor{red}{-}}
\newcommand{\qmark}{\textcolor{orange}{\textbf{?}}}

\title{Related Work}

\author{Tronje Krabbe}

\date{\today}

\begin{document}
\hypersetup{hidelinks}

\setcounter{tocdepth}{1}
\tableofcontents

\chapter{Related Work}
This chapter will give an overview of previous works. It is split into two sections.
The first section presents existing work on the topic of fingerprinting. The second one will show existing work on
the detection of fingerprinting.

\section{Browser Fingerprinting}
There are closed-source implementations of browser/device fingerprinting. These are unavailable to us.
However, we can turn to several open-source projects which implement fingerprinting for deeper insights into this technology,
and to learn how we may be able to identify these closed implementations.

Essentially, fingerprinting works by combining data available to front-end JavaScript code into a string,
usually by hashing some data structures. This string is the fingerprint.
Different fingerprinting methods can be distinguished by the different attributes they use to construct their fingerprint.

For instance, one of the attributes providing the most entropy, is acquired by so-called canvas fingerprinting, which works by drawing
several symbols like geometric shapes, emoji, and so on, on an HTML5 canvas element \cite{laperdrix2016beauty}.
Usually, letters, words, or whole sentences are also written onto the canvas, oftentimes with multiple different fonts.
The resulting image can be serialized into a representation that can be included in the aforementioned fingerprint string.
The fingerprint contains information about the host OS, and the hardware on which it runs.

For a concrete example, \textit{Am I Unique?} writes the same pangram\footnote{A sentence containing every letter of the alphabet.}
twice onto the canvas, using two different fonts, and each time followed by a special unicode character, an emoji.
Finally, a rectangle in a specific color is drawn \cite{laperdrix2016beauty}.

So, while different fingerprinting scripts are all likely to use canvas fingerprinting, they can differ in the exact
method they use to create the canvas-related aspect of the fingerprint.

The methods described in this section can shed some light onto the different attributes used, and their computation,
and how much entropy each can contribute to a fingerprint.

We have compiled an overview of which attributes are used by which implementation in Appendix \ref{app:table}.


\subsection{Am I Unique?}
\textit{Am I Unique?}\footnote{\url{https://amiunique.org}} \cite{laperdrix2016beauty} is a web service which creates
a fingerprint with the press of a button. Its purpose is to educate users about fingerprinting
\textit{Am I Unique?} borrows part of its JavaScript fingerprinting code from
Fingerprintjs2\footnote{\url{https://github.com/DIVERSIFY-project/amiunique/blob/master/website/public/javascripts/webGL.js\#L2}},
which is discussed below.


% FIXME there's a new version of Panopticlick that was released after the paper. Does it include canvas fingerprinting?
\subsection{Panopticlick}
The EFF's Panopticlick project is similar to \textit{Am I Unique?}, in that it aims to learn and inform about
fingerprinting. It also offers a way to test any tracking protection a user may have enabled by simulating
tracking domains \cite{panopticlick}. The attributes used by Panopticlick to construct a fingerprint differ slightly from those used by
\textit{Am I Unique?}, though overall, they are similar.

Eckersley showed that the list of installed fonts and the list of installed plugins are the most identifying properties
\cite{eckersley2010unique}. This finding, however, did not yet take canvas fingerprinting into consideration.


\subsection{Fingerprintjs2}
Fingerprintjs2 \cite{fingerprintjs2} is a popular\footnote{4891 stars and 712 forks on GitHub. Visited on Febuary 4, 2018}
JavaScript program that can be used to easily construct a fingerprint.
It uses a larger variety of attributes when compared to the two previously mentioned services by default, though this is configurable.

A crawl from January 27, 2018, of 4929 sites shows that 336 sites use Fingerprintjs2, for a percentage of about 6.82\%.
This number was found by simply querying the database for function names that are consistent with
those used in Fingerprintjs2, and therefore do not include sites which use an obfuscated Fingerprintjs2.
Fingerprintjs2 has a few functions starting with the prefix \textit{getHasLied}, like \textit{getHasLiedOs}, the purpose
of which is to report whether a browser tampered with certain information, for instance in the interest of mitigating
fingerprinting\footnote{\url{https://github.com/Valve/fingerprintjs2/wiki/Browser-tampering}, visited on Febuary 8, 2018}.
Due to the peculiar name choice, and the fact that the same sites that call one \textit{getHasLied}
function also call all others, we can be quite certain that they all include Fingerprintjs2.


\section{Browser Fingerprinting Detection}
At a basic level, in order to detect fingerprinting, all JavaScript calls to the relevant attributes,
like User Agent, HTML5CanvasElement, and so on, need to be recorded. Then, a decision can be made about whether the number
of calls is suspicious, or constitutes normal website behavior.

However, since these attributes also have legitimate uses, this is not a trivial task.
This section will present previous work on this topic.


\subsection{FPDetective}
Acar et al. provide some useful metrics to differentiate between fingerprinting and legitimate use of
browser attributes \cite{DBLP:conf/ccs/AcarJNDGPP13}. They have implemented a fingerprinting-detection
framework called \textit{FPDetective}. \textit{FPDetective} focuses on font-based fingerprinting, and uses
\textit{PhantomJS} and \textit{Chromium} to crawl the web and log JavaScript and Flash actions.
Popular attributes like the HTML5 Canvas and WebGL were ignored in the study and by the framework's logging.

Partly automated, their process also included manual analysis of scripts and decompiled Flash sources.

\textit{FPDetective} classifies a script as a fingerprinter if ``it loads more than 30 system fonts, enumerates plugins
or mimeTypes, detects screen and navigator properties, and sends the collected data back to a remote server''
\cite{DBLP:conf/ccs/AcarJNDGPP13}.


\subsection{OpenWPM - 1 million site study}
Englehardt et al. have developed a framework called \textit{OpenWPM}, which can be used to analyze the way a website
handles users' privacy. They have used this to analyse the tracking behavior of one million websites.
\cite{DBLP:conf/ccs/EnglehardtN16,englehardt2016census}.

Englehardt et al. differentiate between distinct forms of fingerprinting.
They classify a script as a canvas-fingerprinter, if a there is a canvas larger than 16 by 16 pixels, and text is
written to it in either at least two different colors, or with 10 distinct characters. Then, certain functions
must and must not be called on the object. Their findings show that only about 1.6\% of the analyzed sites
employ canvas-based fingerprinting.

Their work also examines new, previously unknown methods of fingerprinting, such as canvas font fingerprinting, which
works by rendering text in a large number of different fonts on a canvas element; WebRTC-based fingerprinting, which,
as the name suggests, abuses the WebRTC framework; as well as AudioContext fingerprinting, and Battery API fingerprinting.

OpenWPM uses Firefox, Selenium, and custom JavaScript injected into loaded sites to log calls to relevant browser attributes.
It constitutes the technological basis of this thesis' work, as it allows easy crawling of websites, while recording
most if not all relevant information.


\subsection{Tor Browser}
\textit{The Design and Implementation of the Tor Browser [DRAFT]}\footnote{\url{https://www.torproject.org/projects/torbrowser/design/}. Visited on Febuary 8, 2018}
states that ``[the authors] believe that the HTML5 Canvas is the single largest fingerprinting threat browsers face today'',
As a canvas offers easy and high-entropy fingerprinting \cite{acar2014web,mowery2012pixel}.


\begin{appendices}

\chapter{Fingerprinting Attributes}
\label{app:table}

The table below shows which of the presented fingerprinting services/libraries use which attributes of the
browser \cite{am_i_unique,panopticlick,fingerprintjs2}.

\begin{tabular}{ l c c c }
    \arrayrulecolor{hlinegray}

    & Am I Unique? & Panopticlick & Fingerprintjs2 \\
    \hline
    User Agent & \cmark & \cmark & \cmark \\
    \hline
    Accept Header & \cmark & \cmark & \xmark \\
    \hline
    Content Encoding & \cmark & \xmark & \xmark \\
    \hline
    Content Language & \cmark & \xmark & \xmark \\
    \hline
    List of plugins & \cmark & \cmark & \cmark \\
    \hline
    Cookies enabled? & \cmark & \cmark & \xmark \\
    \hline
    Local storage available? & \cmark & \xmark & \cmark \\
    \hline
    Session storage available? & \cmark & \xmark & \cmark \\
    \hline
    Timezone & \cmark & \cmark & \cmark \\
    \hline
    Screen resolution & \cmark & \cmark & \cmark \\
    \hline
    Screen color depth & \cmark & \cmark & \cmark \\
    \hline
    List of fonts & \cmark & \cmark & \cmark \\
    \hline
    List of HTTP headers & \cmark & \xmark & \xmark \\
    \hline
    Platform & \cmark & \cmark & \cmark \\
    \hline
    Do Not Track Header present? & \cmark & \cmark & \cmark \\
    \hline
    Canvas & \cmark & \cmark & \cmark \\
    \hline
    WebGL & \cmark & \cmark & \cmark \\
    \hline
    WebGL Vendor & \cmark & \xmark & \cmark \\
    \hline
    WebGL Renderer & \cmark & \cmark & \cmark \\
    \hline
    Use of an adblocker & \cmark & \xmark & \cmark \\
    \hline
    JavaScript allowed? & \xmark & \cmark & \xmark \\
    \hline
    System Language & \xmark & \cmark & \cmark \\
    \hline
    Touchscreen support & \xmark & \cmark & \cmark \\
    \hline
    IndexedDB available? & \xmark & \xmark & \cmark \\
    \hline
    Open DB? & \xmark & \xmark & \cmark \\
    \hline
    CPU class & \xmark & \xmark & \cmark \\
    \hline
    Available processors & \xmark & \xmark & \cmark \\
    \hline
    Device memory & \xmark & \xmark & \cmark \\
    \hline
\end{tabular}

\end{appendices}

\clearpage

\printbibliography

\end{document}
