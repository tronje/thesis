% rubber: setlist arguments --shell-escape

% ===============================================================================
% = LaTeX Beamer Template des Arbeitsbereichs Sicherheit in verteilten Systemem
% = (c) 2016 Prof. Dr. Hannes Federrath, Uni Hamburg, Fachbereich Informatik
% = https://svs.informatik.uni-hamburg.de
% =
% = Weitgehend in Übereinstimmung mit dem Corporate Design 2016 der UHH:
% = https://www.uni-hamburg.de/beschaeftigtenportal/services/oeffentlichkeitsarbeit/corporate-design.html
% = 
% ===============================================================================
%
\documentclass[t]{beamer} 
% Option t              Place text of slides at the (vertical) top of the slides.
% Option handout        Ein PDF ohne Pausen und Overlayeffekte erzeugen.
% Option aspectratio=43 169 => 16:9, 1610 => 16:10, 43 => 4:3
\usepackage[utf8]{inputenc}
% \usepackage[ngerman]{babel}
\usepackage{graphicx,xcolor}
\usepackage[T1]{fontenc} % 8-Bit-Zeichen; ermöglicht korrektes Kopieren von Umlauten aus dem pdf 
\usepackage{booktabs}
\usepackage{minted}

% SVS-Theme benutzen
\usetheme{svs2016}


% =============================
% = Ab hier Inhalte ändern... 
% =============================

\title[Fingerprinting Detection]{Implementation of browser fingerprinting detection and integration into PrivacyScore}
% \subtitle{Ein Vorschlag}
\author[Krabbe]{Tronje Krabbe}
\institute[Uni Hamburg]{Universität Hamburg\\ Fachbereich Informatik}
\date{12.06.2018}

\begin{document}

\begin{frame}[plain]
	% Die Titelseite erscheit nach erneutem Übersetzen korrekt.
	\maketitle
\end{frame}


\begin{frame}{Agenda}
	% Die Gliederung erscheit nach erneutem Übersetzen korrekt.
	\tableofcontents
\end{frame}

% basic problem; why is it interesting?
\section{Browser Fingerprinting}
\begin{frame}{What is Fingerprinting?}
    \begin{block}{Fingerprinting}
        The creation of a fingerprint, using information about the browser and the
        device it is running on.
    \end{block}

    \pause

    \begin{block}{Fingerprint}
        A string that uniquely identifies a user within a large set of users.
        \pause
        Much like the biological fingerprint of a human.
    \end{block}
\end{frame}

\begin{frame}{Why Fingerprinting?}
    \pause
    \vspace{3cm}
    \centering \LARGE Tracking.
\end{frame}

\begin{frame}{How does it work?}
    Primarily via JavaScript.

    \pause

    \begin{block}{Short answer}
        \begin{enumerate}
            \pause
            \item Gather information about the browser and the device.
            \pause
            \item Create a string, e.g. by hashing these attributes.
            \pause
            \item Done.
        \end{enumerate}
    \end{block}
\end{frame}

\begin{frame}{How does it work?}
    \begin{block}{Long answer}
        Use any and all of these attributes:
        \begin{columns}[T]
            \begin{column}{.4\textwidth}
                \begin{itemize}
                    \item \texttt{userAgent}
                    \item \texttt{language}
                    \item list of installed fonts
                    \item list of plugins
                    \item screen resolution \& color depth
                    \item timezone
                \end{itemize}
            \end{column}
            \begin{column}{.6\textwidth}
                \begin{itemize}
                    \item \texttt{platform}
                    \item \texttt{doNotTrack}
                    \item \texttt{canvas} behavior
                    \item WebGL behavior, renderer, vendor
                    \item Web Audio API
                    \item WebRTC API
                \end{itemize}
            \end{column}
        \end{columns}
    \end{block}
\end{frame}

\begin{frame}{How does it work?}
    \begin{figure}
        \includegraphics[scale=0.6]{pic/canvas_firefox.png}
        \caption{Canvas fingerprinting in Firefox}
    \end{figure}
    \begin{figure}
        \includegraphics[scale=0.5]{pic/canvas_chromium.png}
        \caption{Canvas fingerprinting in Chromium}
    \end{figure}
\end{frame}

% concrete problem tackled by this thesis
\section{Fingerprinting Detection}
\begin{frame}{Why fingerprinting detection?}
    Motivation:
    \begin{itemize}
        \pause
        \item Fingerprints can not be deleted.
        \pause
        \item Fingerprints can be created without the user ever knowing.
    \end{itemize}

    \pause

    Goal: create transparency about the behavior of websites.
\end{frame}

\begin{frame}{Fingerprinting Detection}
    \pause

    Easy!

    \vspace{0.7cm}

    \pause

    \begin{enumerate}
        \item Simply log all JavaScript accesses to fingerprintable attributes!
        \pause
        \item Analyse them to see if a website fingerprints!
        \pause
        \item Done!
    \end{enumerate}

    \pause

    \vspace{0.7cm}

    \textbf{Primary problem:}
    \pause
    When is a website legitimately using a fingerprintable attribute,
    and when is it used for fingerprinting?
\end{frame}

\begin{frame}{How to log JavaScript accesses?}
    OpenWPM\footnote{\url{https://github.com/citp/OpenWPM/}} is the technological basis for this thesis.
    It is also used by PrivacyScore, which came in handy later.
    \vspace{0.5cm}

    \pause

    \begin{block}{OpenWPM}
        OpenWPM uses Firefox, Selenium, and custom JavaScript injected into loaded sites to log
calls to relevant browser attributes.
    \end{block}
    \vspace{0.5cm}

    \pause

    $\Rightarrow$ it is exactly what I needed.
\end{frame}


\section{Methods \& Approach}
\begin{frame}{Methods \& Approach}
    \pause

    \textbf{Gather data:} Have OpenWPM crawl 5000 websites and record their JavaScript calls.
    \vspace{0.3cm}

    \pause

    \textbf{Basic Approach:}

    \pause

    \begin{itemize}
        \item Run automated analyses on the data.
        \pause
        \item Perform manual analyses on the data.
        \pause
        \item Draw conclusions.
        \pause
        \item Use the gained knowledge for the implementation.
    \end{itemize}
\end{frame}

\begin{frame}{Analyses}
    \begin{itemize}
        \item For each of the fingerprintable APIs, find all sites that use them.
        \item Analyse a subset of these manually to learn about fingerprinting behavior.
    \end{itemize}
\end{frame}

\subsection*{Example: Canvas Fingerprinting}
\begin{frame}[fragile]{Example: Canvas Fingerprinting}
    \begin{figure}
    \begin{minted}{sql}
    SELECT DISTINCT script_url
    FROM javascript
    WHERE symbol = 'HTMLCanvasElement.toDataURL';
    \end{minted}
    \caption{SQL query to select all sites using \texttt{toDataURL} on a Canvas.}
    \end{figure}
\end{frame}

\begin{frame}{Example: Canvas Fingerprinting}
    \begin{table}
        \centering
        \caption{\texttt{toDataURL} users}
        \begin{tabular}{l r r}
            \toprule
            & amount & percentage \\
            \midrule
            \textbf{analysed sites} & 75 & 100\% \\
            \midrule
            \textbf{fingerprinters} & 51 & 68\% \\
            \textbf{inconclusive} & 15 & 20\% \\
            \textbf{legitimate} & 9 & 12\% \\
            \bottomrule
        \end{tabular}
    \end{table}

    \vspace{0.5cm}

    \pause
    \textbf{Conclusion:} \texttt{toDataURL} use alone means a high likelihood of fingerprinting.
\end{frame}

\begin{frame}{How to spot fingerprinting}
    \pause

    \begin{itemize}
        \item Search for keywords like ``fingerprint''.
        \pause
        \item Search for magic strings.
        \pause
        \item Search for giveaways that a known fingerprinting script is used.
        \pause
        \item Look for obvious enumeration of attributes and suspicious use of other APIs.
        \pause
        \item Beautify the script, modify and run it to understand what it does.
    \end{itemize}
\end{frame}

\section{Conclusion}
\begin{frame}{Conclusion}
    \begin{columns}[T]
        \begin{column}{0.4\textwidth}
            \textbf{+}
            \begin{itemize}
                \pause
                \item Known fingerprintable APIs are, in fact, very often used for fingerprinting.
                \pause
                \item Fingerprinting detection algorithm has been implemented and integrated into PrivacyScore.
            \end{itemize}
        \end{column}
        \begin{column}{0.6\textwidth}
            \textbf{-}
            \begin{itemize}
                \pause
                \item No APIs that are currently not known to be fingerprintable were explored.
                \pause
                \item Detecting non-canvas font-fingerprinting is not (yet) feasible.
                \pause
                \item Due to OpenWPM limitations, detailed WebGL fingerprinting analysis was not performed.
            \end{itemize}
        \end{column}
    \end{columns}
\end{frame}

\section{Outlook}
\begin{frame}{Outlook / Future Work}
    \begin{itemize}
        \pause
        \item Font Fingerprinting
        \pause
        \item WebGL Fingerprinting
        \pause
        \item Fingerprinting Defense
        \pause
        \item Other APIs
    \end{itemize}
\end{frame}

\begin{frame}[plain]
    \vspace{3.5cm}
    \centering \Large Fin
\end{frame}

\section*{Bonus Slides}
\begin{frame}[plain]
    \vspace{3.5cm}
    \centering \Large Bonus Slides
\end{frame}

\begin{frame}{\texttt{mmmmmmmmmmlli}}
    \framesubtitle{Non-canvas font fingerprinting}

    \pause

    \textbf{Method:}
    \begin{enumerate}
        \item Write some text to an HTML element (like \texttt{span}) in a default font.
        \pause
        \item Measure the element's dimensions.
        \pause
        \item For each font in a large list of fonts:
            \begin{enumerate}
                \item Set the element's font to the current font.
                \item Measure again. If the dimensions are different from the default, the font is installed.
            \end{enumerate}
    \end{enumerate}

    \pause

    \vspace{1cm}
    $\Rightarrow$ add the list of installed fonts to the fingerprint.
\end{frame}

\begin{frame}{Example: Non-Canvas Font Fingerprinting}
    \begin{table}
        \centering
        \caption{\texttt{offsetWidth} and \texttt{offsetHeight} users}
        \begin{tabular}{l r r}
            \toprule
            & amount & percentage \\
            \midrule
            \textbf{analysed sites} & 75 & 100\% \\
            \midrule
            \textbf{fingerprinters} & 5 & 7\% \\
            \textbf{inconclusive} & 7 & 9\% \\
            \textbf{legitimate} & 63 & 84\% \\
            \bottomrule
        \end{tabular}
    \end{table}

    \vspace{0.5cm}

    \pause
    \textbf{Conclusion:} Not useful in fingerprinting detection.
\end{frame}

\end{document}
