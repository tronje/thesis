% rubber: setlist arguments --shell-escape

\documentclass[
    fontsize=12pt,
    headings=small,
    parskip=half,
    bibliography=totoc,
    numbers=noenddot,
    open=any
    ]{scrreprt}

\usepackage[utf8]{inputenc}
% \usepackage{datetime2}
\usepackage[
    autostyle,
    ]{csquotes}
\usepackage{amsmath}
\usepackage[titletoc,title]{appendix}
\usepackage{booktabs}
\usepackage{pifont}
\usepackage{marvosym}
\usepackage{minted}
\usepackage{graphicx}
\usepackage{parskip}
\usepackage{hyperref}
\usepackage{setspace}
\usepackage{acronym}
% \usepackage{float}
\usepackage[scaled=.92]{helvet}
\usepackage{courier}
\usepackage[table]{xcolor}
\usepackage[
    a4paper,
    margin=2.54cm,
    marginparwidth=2.0cm,
    footskip=1.0cm
    ]{geometry}

\usepackage[
    url=true,
    style=numeric,
    alldates=long,
    dateabbrev=false
]{biblatex}
% \usepackage[
%     style=alphabetic,
%     backend=biber,
%     %backref=true
%     ]{biblatex}
\addbibresource{literature.bib}

\clubpenalty=10000

\widowpenalty=10000
\displaywidowpenalty=10000
\usepackage{float}
\floatstyle{plaintop}
\restylefloat{table}
\usepackage{ifdraft}
\pagestyle{plain}
\deffootnote{1em}{1em}{
  \thefootnotemark.\ }

\titlehead{
    \begin{center}
        % \includegraphics[width=4cm]{images/uhh_logo_minimal.png}\\
        \includegraphics[width=8cm]{images/UHH-Logo_2010_Farbe_RGB_hires_nomargin.png}\\
        \medskip
        Department of Computer Science
    \end{center}
}

\newcommand{\dominik}[1]{\textcolor{orange}{\textbf{Notiz: #1}}}
\newcommand{\todo}[1]{\textcolor{red}{\textbf{TODO: #1}}}
\newcommand{\cmark}{\textcolor{green}{\ding{51}}}
% \newcommand{\xmark}{\textcolor{red}{\ding{55}}}
\newcommand{\xmark}{\textcolor{red}{-}}
\newcommand{\qmark}{\textcolor{orange}{\textbf{?}}}

\title{
    Implementation of browser fingerprinting detection \\
    and integration into PrivacyScore
}

\author{Tronje Krabbe}

\date{\today}

\begin{document}
\hypersetup{hidelinks}

% \maketitle

\newpage
\thispagestyle{empty}
% \addcontentsline{toc}{chapter}{Muster des Deckblatts}
\begin{titlepage}% {{{
\includegraphics[width=6.8cm]{images/up-uhh-logo-u-2010-u-farbe-u-rgb.pdf}
\begin{center}\large
	% Universität Hamburg \par
	% Fachbereich Informatik
    \vfill
	\makeatletter
	{\Large\textsf{\textbf{\@title}}\par}
	\makeatother
	\vfill
    A thesis submitted for the degree of \\
    \textit{Bachelor of Science} \\
    in \\
    \textit{Computer Science}
	\vfill
    by
	\par\bigskip
	\makeatletter
	{\@author} \par
	\makeatother
	born 11. August 1994 in Hamburg \par
	matriculation number 6435002
	\vfill
	\makeatletter
	submitted {\@date}
	\makeatother
	\vfill
	Supervisor: Dipl.-Inf. Tobias Müller \par
	First reviewer: Prof. Dr.-Ing. Hannes Federrath \par
	Second reviewer: Prof. Dr. Dominik Herrmann
\end{center}
\end{titlepage}% }}}

\setcounter{tocdepth}{1}
\tableofcontents

%%%%%%%%%%%%%%%%%%%%%%%%%%%%%%%%%%%%%%%%%%%%%%%%%%%%%%%%%%%%%%%%%%%%%%%%%%%%%%
%% Chapter Start                                                            %%
%%%%%%%%%%%%%%%%%%%%%%%%%%%%%%%%%%%%%%%%%%%%%%%%%%%%%%%%%%%%%%%%%%%%%%%%%%%%%%
\chapter{Introduction}
\label{chap:introduction}

When browsing the web, users can be tracked through the use of cookies, which store information on the user's computer
that can also be accessed by a remote server.
If users wish not to be trackable, they can modify or delete any cookie as they see fit.
There are, however, other methods of uniquely identifying users and tracking them during their use of the internet,
which are not as easily mitigated as cookies \cite{am_i_unique}. One of these methods is called ``browser fingerprinting'',
and it will be the focus of this thesis.

Browser fingerprinting, also known as device fingerprinting, and in the following often simply referred to as `fingerprinting',
works by analyzing a web browser's configuration and settings; mostly installed fonts and HTML5 canvas behavior
\cite{DBLP:conf/ccs/EnglehardtN16}, as well as
language settings, time zone settings, installed add-ons, and more. Flash is also sometimes used, though
its end-of-life lies in the foreseeable future, reportedly in the year 2020.
\footnote{\url{https://arstechnica.com/information-technology/2017/07/with-html5-webgl-javascript-ascendant-adobe-to-cease-flash-dev-at-end-of-2020/}}
Fingerprinting through Flash and the detection thereof will thus not be within the scope of this thesis.

The above attributes are readily available to be collected through JavaScript functions. Simply recording
all function calls made by the JavaScript front end of a website can reveal whether fingerprinting is likely to
be taking place or not \cite{faiz2014browser, panopticlick}.

Techniques used to identify and track users across different websites without
their knowledge or their ability to easily intervene, violates their privacy; by creating a fingerprint
of a user's browser, with sufficiently sophisticated techniques, one can uniquely identify one user
among hundreds of thousands \cite{am_i_unique}, re-instate any cookies the user may have deleted,
or simply track and analyze their use of any number of web services for a multitude of malicious reasons \cite{eckersley2010unique}.
Methods exist to mitigate the effectiveness of browser fingerprinting, such as the one presented
in \cite{laperdrix2015mitigating};
these are, however, usually much more complicated and specialized for the average user of a web-browser.
The author has therefore implemented a technique to recognize when a website is deploying browser fingerprinting,
and along the way explored the techniques used to do so. The resulting software has been integrated into
\url{https://privacyscore.org}, a web-service to test and rank websites according to the extent to which they
respect their users' privacy; this creates a different kind of defense against the privacy violating
methods that form the topic of this thesis: informing users about websites they use and their treatment of
sensitive information.

\section{Motivation}
There is a fair amount of related work on the topic. However, since it exists in the fields of privacy and security,
it always deals with some form of `adversary', meaning something or someone that can change and adapt to new
research and methods. It is therefore important to keep up with any changes, which can happen very quickly
in the context of software.
In this case, someone can potentially come up with some new, not-yet-detectable fingerprinting mechanism,
requiring new research to be done again, creating motivation from a technical perspective.

Tracking a user does not simply mean recognizing whether a visitor to one's website is a new or an existing user.
Identifying information can be passed on or sold to partners, advertisers, or even governments, in order to
construct a rich browsing history of a user. Users are required by EU law to be informed of cookies being set
by websites CITATION NEEDED, yet fingerprinting can and does happen quietly.
The work of this thesis aims to create the possibility of users at least informing themselves about the practices
of the websites they visit, giving a social motivational aspect.

\section{Fundamentals}
The following section will explain some fundamental knowledge that is
referenced throughout this thesis.
\dominik{ist diese section cool oder soll sie weg?}

\subsection{Fingerprint}
\label{fundamentals:fingerprint}
Throughout the thesis, terms such as ``fingerprint'', ``browser fingerprint'', etc.
will be used largely interchangeably.
A fingerprint, in most contexts, is something that can uniquely identify something else.
The obvious example is a human's fingerprint, which is unique to each finger of each human,
with very few exceptions, if any.\footnote{\url{https://www.interpol.int/INTERPOL-expertise/Forensics/Fingerprints}}

\subsection{Entropy}
\label{fundamentals:entropy}
Entropy is the most important aspect of a fingerprint.
Peter Eckersley describes entropy as
``a mathematical quantity which allows us to measure how close a fact comes to revealing somebody's identity uniquely''.\footnote{\url{https://www.eff.org/deeplinks/2010/01/primer-information-theory-and-privacy}}
Entropy is often measured in bits, because it measures the different
possible values something can take.
To give an example, the outcome of a coin toss has one bit of entropy,
as it can have two different values: heads or tails.
A random, unknown human's identity contains about 33 bits of entropy,
as two to the power of 33 is 8 billion, and there are some 7.6 billion humans
alive at the time of writing.
If a browser fingerprint can provide 33 bits of information, it can be used
to uniquely identify any person.

\subsection{JavaScript}
\label{fundamentals:javascript}
JavaScript\footnote{\url{https://developer.mozilla.org/en-US/docs/Web/JavaScript}} is,
at the time of writing, the primary programming languages understood
by modern web browsers. While HTML and CSS dictate the visual layout and style
of a web page, JavaScript provides its functionality.
JavaScript is important in the context of this thesis, because
it can be used to construct a browser fingerprint. JavaScript can query
a browser's or device's attributes, transform them, and transmit them
to a third party.
Within this thesis, the JavaScript code executed by websites
when visited, will be saved and analyzed.
This constitutes the primary source of data for the thesis.

\subsection{Canvas}
\label{fundamentals:canvas}
The canvas is a HTML5 element, upon which a web page can draw,
using one of several methods. For instance, WebGL allows rendering of complex
3D graphics on a canvas.
The canvas can also be manipulated by JavaScript. For this thesis,
the canvas plays an important role. It enables the generation
of images with high entropy values in regards to their fingerprinting value.
We will show that many websites draw on a canvas, extract the resultant raw pixel
values from the canvas, and use it to contribute to the generation of a fingerprint.
This is possible for a number of reasons. For one, a JavaScript program
can instruct the browser it is running in to draw emoji on a canvas.
If the browser supports emoji, they will be drawn a certain way, usually dictated
by the device the browser is running on. Apple's emoji look different from Google's
emoji, for instance.\footnote{\url{https://emojipedia.org/apple/}}\footnote{\url{https://emojipedia.org/google/}}
If emoji are not supported, fallback behavior often means that black squares will
be drawn in their place.
Thus, drawing simple emoji on a canvas can give up information about a device's vendor.
A similar approach can be taken with fonts.
Creating a WebGL rendering context and rendering images on it can reveal
information about the GPU vendor of the host device.

\todo{noch mehr?}

\section{Leading Question and Goals}
\dominik{fand ich nicht so gut, hab ich erstmal rausgenommen}
% The overall goal of this thesis is to create a software that can reliably report whether a website carries out
% fingerprinting, in order to expose these websites for using privacy-violating techniques, and integrating it into
% \url{https://privacyscore.org}.
% This goal is easily stated, but with it come several questions which will need to be answered before
% it can be reached.

% \subsection{What are the most commonly used fingerprinting methodologies?}
% Methods to fingerprint change. Flash, for example, can be used to construct a fingerprint, but is expected
% to be used less and less, since its end of life has been announced.
% So, what are the most-used methods \textit{today}, and how much will these change over time? Which methods
% can be expected to still be used in a year, and which will be deprecated soon?
%
% \subsection{What are the most commonly used fingerprinting libraries and how can they be found?}
% Perhaps the most used library is a closed-source, proprietary product by some large conglomerate,
% with no publicly available information. If its use can be reliably recognized, many websites could be found guilty
% almost immediately. This, however, isn't a given, and it may take considerable effort to answer this question.
%
% \subsection{How can false positives be minimized?}
% False positives, i.e. websites that are identified as fingerprinters, but don't actually employ any such technology,
% are detrimental to the overall results of the software. How can they be minimized or, ideally, eliminated?
%
% \subsection{What does a good/fair metric look like?}
% The impact of false positives can be diminished by employing a rating system or metric to rate a website's
% ``fingerprinting score''. Instead of reporting boolean results (true or false), the software should
% report a number or score that represents the likelihood of fingerprinting taking place, as determined by the software.
% How is such a score calculated, and which attributes are used in this calculation?
%
% \subsection{Can all this be achieved with acceptable performance?}
% This is as much a leading question as it is a goal: the software \textit{must} exhibit adequate performance.
% Since a key component of the stated goal is to integrate the software into \url{https://privacyscore.org},
% it must be able to analyze a website within seconds, to allow acceptable performance of PrivacyScore itself.
% If the software were designed to simply be run once for some large number of websites, so that the results
% could be displayed, it would be fine if it took some arbitrary amount of time. But in order to be useful
% in the desired context, it is imperative that performance is at least something to be kept in mind.

\section{Organization}
The thesis' structure is shown in the following.
In Chapter~\ref{chap:related_work}, we present related work, split into two main groups:
related work on the topic of browser fingerprinting, and related work on
browser fingerprinting detection.
Chapter~\ref{chap:data_acquisition}, will showcase our data acquisition
methods, and discuss some basic analyses.
\todo{alle chapter erklären}

% \section{Methods and Approach}

%%%%%%%%%%%%%%%%%%%%%%%%%%%%%%%%%%%%%%%%%%%%%%%%%%%%%%%%%%%%%%%%%%%%%%%%%%%%%%
%% Chapter End                                                              %%
%%%%%%%%%%%%%%%%%%%%%%%%%%%%%%%%%%%%%%%%%%%%%%%%%%%%%%%%%%%%%%%%%%%%%%%%%%%%%%


%%%%%%%%%%%%%%%%%%%%%%%%%%%%%%%%%%%%%%%%%%%%%%%%%%%%%%%%%%%%%%%%%%%%%%%%%%%%%%
%% Chapter Start                                                            %%
%%%%%%%%%%%%%%%%%%%%%%%%%%%%%%%%%%%%%%%%%%%%%%%%%%%%%%%%%%%%%%%%%%%%%%%%%%%%%%
\chapter{Related Work}
\label{chap:related_work}
This chapter will give an overview of previous works. It is split into two sections.
The first section presents existing work on the topic of fingerprinting. The second one will show existing work on
the detection of fingerprinting.

\section{Browser Fingerprinting}
There are closed-source implementations of browser/device fingerprinting. These are unavailable to us.
However, we can turn to several open-source projects which implement fingerprinting for deeper insights into this technology,
and to learn how we may be able to identify these closed implementations.

Essentially, fingerprinting works by combining data available to front-end JavaScript code into a string,
usually by hashing some data structures. This string is the fingerprint.
Different fingerprinting methods can be distinguished by the different attributes they use to construct their fingerprint.

For instance, one of the attributes providing the most entropy, is acquired by so-called canvas fingerprinting, which works by drawing
several symbols like geometric shapes, emoji, and so on, on an HTML5 canvas element \cite{laperdrix2016beauty}.
Usually, letters, words, or whole sentences are also written onto the canvas, oftentimes with multiple different fonts.
The resulting image can be serialized into a representation that can be included in the aforementioned fingerprint string.
The fingerprint contains information about the host OS, and the hardware on which it runs.

For a concrete example, \textit{Am I Unique?} writes the same pangram\footnote{A sentence containing every letter of the alphabet.}
twice onto the canvas, using two different fonts, and each time followed by a special Unicode character, an emoji.
Finally, a rectangle in a specific color is drawn \cite{laperdrix2016beauty}.

So, while different fingerprinting scripts are all likely to use canvas fingerprinting, they can differ in the exact
method they use to create the canvas-related aspect of the fingerprint.

The methods described in this section can shed some light onto the different attributes used, and their computation,
and how much entropy each can contribute to a fingerprint.

We have compiled an overview of which attributes are used by which implementation in Table~\ref{app:attribute_table}.


\subsection{Am I Unique?}
\label{related_work:am_i_unique}
\textit{Am I Unique?} \cite{laperdrix2016beauty} is a web service which creates
a fingerprint with the press of a button.\footnote{\url{https://amiunique.org}}
Its purpose is to educate users about fingerprinting
\textit{Am I Unique?} borrows part of its JavaScript fingerprinting code from
Fingerprintjs2, which is discussed below.\footnote{\url{https://github.com/DIVERSIFY-project/amiunique/blob/master/website/public/javascripts/webGL.js\#L2}}


% FIXME there's a new version of Panopticlick that was released after the paper. Does it include canvas fingerprinting?
\subsection{Panopticlick}
\label{related_work:panopticlick}
The EFF's Panopticlick project is similar to \textit{Am I Unique?}, in that it aims to learn and inform about
fingerprinting. It also offers a way to test any tracking protection a user may have enabled by simulating
tracking domains \cite{panopticlick}. The attributes used by Panopticlick to construct a fingerprint differ slightly from those used by
\textit{Am I Unique?}, though overall, they are similar.

Eckersley showed that the list of installed fonts and the list of installed plugins are the most identifying properties
\cite{eckersley2010unique}. This finding, however, did not yet take canvas fingerprinting into consideration.


\subsection{Fingerprintjs2}
\label{related_work:fingerprintjs2}
Fingerprintjs2 \cite{fingerprintjs2} is a popular JavaScript program that
can be used to easily construct a fingerprint.\footnote{4891 stars and 712 forks on GitHub. Visited on February 4, 2018}
It uses a larger variety of attributes when compared to the two previously mentioned services by default, though this is configurable.

A crawl from January 27, 2018, of 4929 sites shows that 336 sites use Fingerprintjs2, for a percentage of about 6.82\%.
This number was found by simply querying the database for function names that are consistent with
those used in Fingerprintjs2, and therefore do not include sites which use an obfuscated Fingerprintjs2.
Fingerprintjs2 has a few functions starting with the prefix \textit{getHasLied}, like \textit{getHasLiedOs}, the purpose
of which is to report whether a browser tampered with certain information, for instance in the interest of mitigating
fingerprinting.\footnote{\url{https://github.com/Valve/fingerprintjs2/wiki/Browser-tampering} Visited on February 8, 2018}
Due to the peculiar name choice, and the fact that the same sites that call one \textit{getHasLied}
function also call all others, we can be quite certain that they all include Fingerprintjs2.


\section{Browser Fingerprinting Detection}
At a basic level, in order to detect fingerprinting, all JavaScript calls to the relevant attributes,
like User Agent, HTML5CanvasElement, and so on, need to be recorded. Then, a decision can be made about whether the number
of calls is suspicious, or constitutes normal website behavior.

However, since these attributes also have legitimate uses, this is not a trivial task.
This section will present previous work on this topic.


\subsection{FPDetective}
\label{related_work:fpdetective}
Acar et al. provide some useful metrics to differentiate between fingerprinting and legitimate use of
browser attributes \cite{DBLP:conf/ccs/AcarJNDGPP13}. They have implemented a fingerprinting-detection
framework called \textit{FPDetective}.
\textit{FPDetective} focuses on font-based fingerprinting, and uses
\textit{PhantomJS} to log JavaScript function calls and \textit{Chromium} to log Flash actions.
Much like OpenWPM, their crawler logs accesses to certain browser and device properties, such as
those contained within the \textit{window.navigator} object.
Popular attributes like the HTML5 Canvas and WebGL were ignored in the study and by the framework's logging.

Partly automated, their process consisted of an automated analysis in regards to font fingerprinting,
and a manual analysis of JavaScript and decompiled Flash code.
Their focus on fonts derives from the assumption that a website that
enumerates fonts is a likely fingerprinter.

\textit{FPDetective} classifies a script as a fingerprinter if ``it loads more than 30 system fonts, enumerates plugins
or mimeTypes, detects screen and navigator properties, and sends the collected data back to a remote server''
\cite{DBLP:conf/ccs/AcarJNDGPP13}.
\dominik{Das hier hattest du markiert mit ``sounds good. you could use the same approach''. Also sowas dazu schreiben?} \\ \\
This is a promising metric, and we could use the same approach. \\ \\
\dominik{oder lieber} \\ \\
This promising metric will be taken into account in our implementation. \\ \\
\dominik{oder} \\ \\
In Chapter~\ref{todo}, we show that this metric is \todo{accurate/useful or not?}.

As Flash is nearing its end-of-life, this thesis opts not to analyze Flash programs
delivered by websites, which means \textit{FPDetective}'s extensive work on Flash-based fingerprinting
is, of little use for us.


\subsection{OpenWPM - 1 million site study}
\label{section:openwpm}
Englehardt et al. have developed a framework called \textit{OpenWPM}, which can be used to analyze the way a website
handles users' privacy. They have used this to analyze the tracking behavior of one million websites.
\cite{DBLP:conf/ccs/EnglehardtN16,englehardt2016census}.

Englehardt et al. differentiate between distinct forms of fingerprinting.
They classify a script as a canvas-fingerprinter, if a there is a canvas larger than 16 by 16 pixels, and text is
written to it in either at least two different colors, or with 10 distinct characters.
A canvas-fingerprinter script should not call the \textit{save}, \textit{restore} or \textit{addEventListener}
methods of the rendering context. Finally, the script has to call either \textit{toDataURL}
or \textit{getImageData} for an area of at least 16 pixels in width and height.
They have manually verified the accuracy of their heuristic, and found only four false positives
out of 3493 scripts.
We will show that this heuristic is \todo{good/not good?}.

Englehardt et al. make a distinction between canvas fingerprinting and canvas font fingerprinting.
They classify a script as a canvas font fingerprinter if it sets the \textit{font} property
of a canvas at least 50 times, and also calls the \textit{measureText} method at least 50 times.
They have manually verified that there were no false positives for this metric.

Their findings show that only about 1.6\% of the analyzed sites
employ canvas-based fingerprinting. However, they also show that among the top 1.000 sites,
5.1\% of sites employ canvas-based fingerprinting.

Their work also examines new, previously unknown methods of fingerprinting, such as canvas font fingerprinting, which
works by rendering text in a large number of different fonts on a canvas element; WebRTC-based fingerprinting, which,
as the name suggests, abuses the WebRTC framework; as well as AudioContext fingerprinting, and Battery API fingerprinting.

OpenWPM uses Firefox, Selenium, and custom JavaScript injected into loaded sites to log calls to relevant browser attributes.
It constitutes the technological basis of this thesis' work, as it allows easy crawling of websites, while recording
most if not all relevant information.


\subsection{Tor Browser}
\label{related_work:tor_browser}
\textit{The Design and Implementation of the Tor Browser [DRAFT]}
states that ``[the authors] believe that the HTML5 Canvas is the single largest fingerprinting threat browsers face today'',
As a canvas offers easy and high-entropy fingerprinting
\cite{acar2014web,mowery2012pixel}.\footnote{\url{https://www.torproject.org/projects/torbrowser/design/} Visited on February 8, 2018}
The Tor Browser can alert users to data collection from a HTML5 Canvas element.
Data collection, in this context, means using a function like \textit{toDataURL} on the element.

%%%%%%%%%%%%%%%%%%%%%%%%%%%%%%%%%%%%%%%%%%%%%%%%%%%%%%%%%%%%%%%%%%%%%%%%%%%%%%
%% Chapter End                                                              %%
%%%%%%%%%%%%%%%%%%%%%%%%%%%%%%%%%%%%%%%%%%%%%%%%%%%%%%%%%%%%%%%%%%%%%%%%%%%%%%


%%%%%%%%%%%%%%%%%%%%%%%%%%%%%%%%%%%%%%%%%%%%%%%%%%%%%%%%%%%%%%%%%%%%%%%%%%%%%%
%% Chapter Start                                                            %%
%%%%%%%%%%%%%%%%%%%%%%%%%%%%%%%%%%%%%%%%%%%%%%%%%%%%%%%%%%%%%%%%%%%%%%%%%%%%%%
\chapter{Data Acquisition and Analysis}
\label{chap:data_acquisition}
This chapter presents the data set used to conduct the experiments used in this thesis.
We will describe how the data was acquired, as well as some basic properties of the set.

\section{Acquiring Site Information}
As described in section~\ref{section:openwpm}, we have used OpenWPM\footnote{\url{https://github.com/citp/OpenWPM}} to gather data about
website behavior.
The parameters used with OpenWPM to gather the data can be found in Appendix~\ref{app:params}.

OpenWPM saves most data in SQLite3 databases.\footnote{\url{https://www.sqlite.org/index.html}}
This includes the data interesting to us: JavaScript calls, function names, and script URLs.
Python3's standard library makes it possible to interact with SQLite3
databases.\footnote{\url{https://docs.python.org/3/library/sqlite3.html}}
Thus, all of our analyses are implemented in Python3.
Physical distributions of this thesis include the source code on a CD-ROM.
\dominik{Sollte das mit der CD erwähnt werden?}
We will show various methods used in our analyses in Chapter \todo{Kapitel schreiben und dann hier ref'en}

Alexa regularly compiles a list of the top one million websites, ordered by the number of
visitors\footnote{\url{https://www.alexa.com/topsites}}.
We have analyzed the top 5000 websites, as described by a version of this list acquired from
\url{http://s3.amazonaws.com/alexa-static/top-1m.csv.zip} on December 19, 2017.

Multiple data sets based on this list were gathered. Due to timeouts or other
errors, not all 5000 sites were actually crawled each time.
The primary data set we used was crawled on January 27, 2018, and includes 4929 sites with distinct URLs. Some of these sites
may be duplicates of each other. For instance, the URL \url{https://fbcdn.net/} for Facebook's content delivery
network simply points to \url{https://facebook.com/}, both of which are included in the data set.

We make the distinction between the primary data set and secondary data sets.
We have run all analyses on the primary data set, and we will show
that the noted behavior is consistent by repeating some analyses
on the secondary sets.
We do this because running all analyses on all sets is beyond the time
constraints of this work.
In the following, we will refer to the primary data set as simply ``the data set''.

\section{Data Properties}
In this section, we will give an overview of basic properties of the primary data set,
with special regard to their relevance for fingerprinting.
Table~\ref{table:dataprops} shows some of these properties.

\renewcommand{\arraystretch}{1.2}
\begin{table}
\centering
\caption{Analyzed Website's Behavior}
\begin{tabular}{l r r}
    \toprule
% \arrayrulecolor{hlinegray}
    & amount & percentage \\
    All sites & 4929 & 100\% \\
    \midrule
    \textit{basic attribute look ups} & & \\
    \textit{userAgent} & 4424 & 90\% \\ %89.8
    \textit{language} or \textit{languages} & 3985 & 81\% \\ %80.8
    \textit{colorDepth} & 3931 & 80\% \\ %79.8
    \textit{localStorage} & 3772 & 77\% \\
    \textit{platform} & 3275 & 66\% \\ %66.4
    \textit{sessionStorage} & 2860 & 58\% \\
    \textit{cookieEnabled} & 2823 & 57\% \\ %57.3
    \textit{doNotTrack} & 1406 & 29\% \\ %28.5
    \textit{oscpu} & 580 & 12\% \\ %11.8
    \midrule
    \textit{using multiple of above attributes} & & \\
    at least 2 & 4338 & 88\% \\
    at least 3 & 4245 & 86\% \\
    at least 4 & 3968 & 81\% \\
    at least 5 & 3572 & 72\% \\
    at least 6 & 2906 & 59\% \\
    at least 7 & 1963 & 40\% \\
    at least 8 & 1138 & 23\% \\
    9 attributes & 473 & 10\% \\
    \midrule
    \textit{Canvas-related functions or attributes} & & \\
    Using any canvas-related JavaScript & 1657 & 34\% \\ %33.6
    \textit{toDataURL} or \textit{getImageData} & 884 & 18\% \\ %17.9
    \textit{toDataURL} & 768 & 16\% \\ %15.6
    \textit{getImageData} & 252 & 5\% \\ %5.1
    \midrule
    Calling functions defined by \textit{Fingerprintjs2} & 336 & 7\% \\ %6.8
    \bottomrule
\end{tabular}
\label{table:dataprops}
\end{table}

The first block of Table~\ref{table:dataprops}, titled \textit{basic attribute look ups},
shows how many sites looked up certain browser- or device attributes.
We label these as `basic', because a JavaScript program can access them with a single function
call. In this section of the table, we can see that, for example, 90\% of sites in the set access the
\textit{userAgent} property.
How many websites access which attribute is interesting to us because of the basic attributes' ease of access.
As their values are easy to acquire, one might expect that these attributes are consistently used by different fingerprinters.

The next section in the table shows how many sites use more than one of the basic attributes.
Note that the rows are not mutually exclusive. This means that, e.g. the 4245 sites using ``at least'' three
attributes also contain all sites that use more than three attributes.
This information is useful, because fingerprints are and need to be constructed from
multiple data points. If they weren't, they would not be unique enough.
\todo{Welche Quelle hier citen?}

The final block of Table~\ref{table:dataprops} is focused on the HTML5 Canvas element.
More specifically, we show how many sites use any kind of Canvas-related JavaScript,
as well as how many use the specific data extraction methods.
A website qualifies as using ``any Canvas-related JavaScript'' when it makes use
of any function calls that are part of the Canvas API\footnote{\url{https://developer.mozilla.org/en-US/docs/Web/API/Canvas_API}}.
The two methods that can be used for data extraction are
\textit{toDataURL}\footnote{\url{https://developer.mozilla.org/en-US/docs/Web/API/HTMLCanvasElement/toDataURL}}
and
\textit{getImageData}\footnote{\url{{https://developer.mozilla.org/en-US/docs/Web/API/CanvasRenderingContext2D/getImageData}}}.
This part of the table is interesting to us, as the Canvas element is one of the
most useful elements for the construction of a fingerprint, as described
in Section~\ref{fundamentals:canvas}.

Of the 884 sites that use \texttt{toDataURL} or \texttt{getImageData} on a HTML5 Canvas object, which suggests
a high likelihood of canvas fingerprinting, 536, or about 61\%, also use eight or more of the basic attributes.
These 536 sites (not shown in the table) are good candidates for likely fingerprinters.
And indeed, 147 of them are contained in the 336 sites that we have identified as users of \textit{Fingerprintjs2}.
\textit{Fingerprintjs2}, as discussed in Section~\ref{}, can be configured to include any number of attributes
in its generated fingerprint. From the above, we can also conclude that 189 users of \textit{Fingerprintjs2}
do not use either of the data extraction methods of the Canvas element. This is curious because
of the Canvas' high entropy potential.
In Section \todo{}, we will show that this is \todo{not?} an anomaly in the data we have collected.

%%%%%%%%%%%%%%%%%%%%%%%%%%%%%%%%%%%%%%%%%%%%%%%%%%%%%%%%%%%%%%%%%%%%%%%%%%%%%%
%% Chapter End                                                              %%
%%%%%%%%%%%%%%%%%%%%%%%%%%%%%%%%%%%%%%%%%%%%%%%%%%%%%%%%%%%%%%%%%%%%%%%%%%%%%%


%%%%%%%%%%%%%%%%%%%%%%%%%%%%%%%%%%%%%%%%%%%%%%%%%%%%%%%%%%%%%%%%%%%%%%%%%%%%%%
%% Chapter Start                                                            %%
%%%%%%%%%%%%%%%%%%%%%%%%%%%%%%%%%%%%%%%%%%%%%%%%%%%%%%%%%%%%%%%%%%%%%%%%%%%%%%
\chapter{Design and Implementation}
Is there enough to be said about design and implementation to warrant an entire chapter?

%%%%%%%%%%%%%%%%%%%%%%%%%%%%%%%%%%%%%%%%%%%%%%%%%%%%%%%%%%%%%%%%%%%%%%%%%%%%%%
%% Chapter End                                                              %%
%%%%%%%%%%%%%%%%%%%%%%%%%%%%%%%%%%%%%%%%%%%%%%%%%%%%%%%%%%%%%%%%%%%%%%%%%%%%%%

%%%%%%%%%%%%%%%%%%%%%%%%%%%%%%%%%%%%%%%%%%%%%%%%%%%%%%%%%%%%%%%%%%%%%%%%%%%%%%
%% Chapter Start                                                            %%
%%%%%%%%%%%%%%%%%%%%%%%%%%%%%%%%%%%%%%%%%%%%%%%%%%%%%%%%%%%%%%%%%%%%%%%%%%%%%%
\chapter{Evaluation}
Evaluation has been performed, and it was great.

%%%%%%%%%%%%%%%%%%%%%%%%%%%%%%%%%%%%%%%%%%%%%%%%%%%%%%%%%%%%%%%%%%%%%%%%%%%%%%
%% Chapter End                                                              %%
%%%%%%%%%%%%%%%%%%%%%%%%%%%%%%%%%%%%%%%%%%%%%%%%%%%%%%%%%%%%%%%%%%%%%%%%%%%%%%

%%%%%%%%%%%%%%%%%%%%%%%%%%%%%%%%%%%%%%%%%%%%%%%%%%%%%%%%%%%%%%%%%%%%%%%%%%%%%%
%% Chapter Start                                                            %%
%%%%%%%%%%%%%%%%%%%%%%%%%%%%%%%%%%%%%%%%%%%%%%%%%%%%%%%%%%%%%%%%%%%%%%%%%%%%%%
\chapter{Conclusion and Outlook}
This thesis was terrific and there's lots of cool stuff to look out for.

%%%%%%%%%%%%%%%%%%%%%%%%%%%%%%%%%%%%%%%%%%%%%%%%%%%%%%%%%%%%%%%%%%%%%%%%%%%%%%
%% Chapter End                                                              %%
%%%%%%%%%%%%%%%%%%%%%%%%%%%%%%%%%%%%%%%%%%%%%%%%%%%%%%%%%%%%%%%%%%%%%%%%%%%%%%

% }}



% \begingroup
% \renewcommand{\cleardoublepage}{}
% \renewcommand{\clearpage}{}
% \chapter{Methods and Approach} % {{
% \endgroup
%
% \section{False Positives}
% Filtering false positives will prove to be the most challenging problem in this context.
% Requesting fonts via JavaScript might be a dead giveaway
% that fingerprinting code is, in fact, fingerprinting code. However, it might cause legitimate code, such
% as a multimedia player, to be mistaken for fingerprinting code, as well, even though it has a legitimate
% reason and perhaps a necessity to know a system's installed fonts.
% Meta-data can and possibly must be used to filter these; a list can be compiled of popular JavaScript libraries with legitimate,
% yet fingerprinting-like behavior, and then extended through analysis of collected data (see Technology below).
%
% All results must be evaluated regarding both precision and recall. Suppose a fingerprinting-recognition software
% analyzes 100 websites. It reports that 50 of them employ fingerprinting, and that the other 50 do not.
% In reality, however, only 40 of the 50 ``positives'' actually do employ fingerprinting, but a total
% of 80 of all the analyzed sites employ it.
% The precision of this software would then be $40/50 = 4/5$, as 40 out of 50 of its reported positives were correct.
% Its recall would be $40/80 = 1/2$, as it only reported 40 positives out of 80 actual positives.
% One might say that precision describes the usefulness of the results, while recall describes their completeness.
%
% \section{Metric}
% If a website includes a well-known browser fingerprinting library, or makes many calls to JavaScript functions that return
% data which is useful in building a fingerprint, one can conclude: this website is very likely generating,
% saving, and possibly distributing a browser's fingerprint; simply put: it performs browser fingerprinting.
% However, some modern websites may exhibit some similar traits or behavior that can be used to construct a fingerprint
% for reasons other than identification. Analyzing installed fonts has an obvious, legitimate use: correctly displaying
% text. Thus, it is important to work out a metric or rating system of some sort, which can
% (somewhat) reliably represent the likelihood that fingerprinting is, in fact, taking place.
% Acar et al. assert that the more fonts are being requested, the more likely a website is to
% be practicing fingerprinting \cite{DBLP:conf/ccs/AcarJNDGPP13}.
% This is based on findings of Eckersley, of the Panopticlick project \cite{eckersley2010unique}.
% Acar et al. further state that they classify a JavaScript file as a fingerprinter ``when it loads
% more than 30 fonts, enumerates plugins or mimeTypes, detects screen and navigator properties, and sends the
% collected data back to a remote server'' \cite{DBLP:conf/ccs/AcarJNDGPP13}.
% Englehardt and Narayanan present a metric to detect Canvas-based fingerprinting by specifying certain
% attributes such a Canvas, and the code using it, should have \cite{DBLP:conf/ccs/EnglehardtN16}.
% The above considerations form a good basis upon which to build a reliable metric, which will have to be
% evaluated further during the implementation process.
%
% \section{Scope}
% Finding or creating a single website which collects a fingerprint, and basing development of fingerprinting
% recognition on it, will not yield reliable or representative results. It is imperative to test
% the created recognition software against a multitude of fingerprinting libraries and users of these.
% It will be prudent to test the software against, say, the top 500 websites as ranked by
% Alexa\footnote{\url{http://www.alexa.com/topsites}}, as there is a high likelihood these sites employ
% a multitude of techniques to track their users, but there is no way to be sure if some of these sites wouldn't
% generate false positives.
% The safest way would be to create websites which employ an array of fingerprinting libraries both preexisting and
% implemented by the author, and then comparing the employed JavaScript calls with those from popular websites.
% The best approach seems to be a combination of the above; create a set of fake websites, some of which employ fingerprinting,
% others mimicking it. Then the results from analyzing them will be compared to the results from analyzing top Alexa sites.
%
% % }}

\begin{appendices}

\chapter{Fingerprinting Attributes}

Table~\ref{app:attribute_table} shows which of the presented fingerprinting services/libraries use which attributes of the
browser \cite{am_i_unique,panopticlick,fingerprintjs2}.

\dominik{Wirklich keine hlines in der table? damit man besser erkennen kann, welches Häkchen wozu gehört?}

\renewcommand{\arraystretch}{1.2}
\begin{table}
\centering
\caption{Different Browser Attributes used by Fingerprinting Projects}
\begin{tabular}{ l c c c }
    \toprule
    & Am I Unique? & Panopticlick & Fingerprintjs2 \\
    User Agent & \cmark & \cmark & \cmark \\
    Accept Header & \cmark & \cmark & \xmark \\
    Content Encoding & \cmark & \xmark & \xmark \\
    Content Language & \cmark & \xmark & \xmark \\
    List of plugins & \cmark & \cmark & \cmark \\
    Cookies enabled? & \cmark & \cmark & \xmark \\
    Local storage available? & \cmark & \xmark & \cmark \\
    Session storage available? & \cmark & \xmark & \cmark \\
    Timezone & \cmark & \cmark & \cmark \\
    Screen resolution & \cmark & \cmark & \cmark \\
    Screen color depth & \cmark & \cmark & \cmark \\
    List of fonts & \cmark & \cmark & \cmark \\
    List of HTTP headers & \cmark & \xmark & \xmark \\
    Platform & \cmark & \cmark & \cmark \\
    Do Not Track Header present? & \cmark & \cmark & \cmark \\
    Canvas & \cmark & \cmark & \cmark \\
    WebGL & \cmark & \cmark & \cmark \\
    WebGL Vendor & \cmark & \xmark & \cmark \\
    WebGL Renderer & \cmark & \cmark & \cmark \\
    Use of an ad blocker & \cmark & \xmark & \cmark \\
    JavaScript allowed? & \xmark & \cmark & \xmark \\
    System Language & \xmark & \cmark & \cmark \\
    Touchscreen support & \xmark & \cmark & \cmark \\
    IndexedDB available? & \xmark & \xmark & \cmark \\
    Open DB? & \xmark & \xmark & \cmark \\
    CPU class & \xmark & \xmark & \cmark \\
    Available processors & \xmark & \xmark & \cmark \\
    Device memory & \xmark & \xmark & \cmark \\
    \bottomrule
\end{tabular}
\label{app:attribute_table}
\end{table}


\chapter{OpenWPM Browser Params}
The following configuration for OpenWPM's \textit{default\_browser\_params} was used.

\definecolor{mintedbg}{rgb}{0.95,0.95,0.95}
\setminted{fontsize=\footnotesize,bgcolor=mintedbg}

\begin{figure}
\label{app:params}
\begin{minted}{json}
{
    "extension_enabled": true,
    "cookie_instrument": false,
    "js_instrument": true,
    "cp_instrument": true,
    "http_instrument": true,
    "save_javascript": true,
    "save_all_content": false,

    "random_attributes": false,
    "bot_mitigation": false,
    "disable_flash": true,
    "profile_tar": null,
    "profile_archive_dir": null,
    "headless": true,
    "browser": "firefox",
    "prefs": {},

    "tp_cookies": "always",
    "donottrack": false,
    "disconnect": false,
    "ghostery": false,
    "https-everywhere": false,
    "adblock-plus": false,
    "ublock-origin": false,
    "tracking-protection": false
}
\end{minted}
\caption{The Browser Parameters used by OpenWPM in our Crawls}
\end{figure}

\end{appendices}

\clearpage

\printbibliography

\end{document}
