% rubber: setlist arguments --shell-escape

\documentclass[
    fontsize=12pt,
    headings=small,
    parskip=half,
    bibliography=totoc,
    numbers=noenddot,
    open=any
    ]{scrreprt}

\usepackage[utf8]{inputenc}
% \usepackage{datetime2}
\usepackage[
    autostyle,
    ]{csquotes}
\usepackage{amsmath}
\usepackage[titletoc,title]{appendix}
\usepackage{booktabs}
\usepackage{pifont}
\usepackage{marvosym}
\usepackage{minted}
\usepackage{graphicx}
\usepackage{parskip}
\usepackage{hyperref}
\usepackage{setspace}
\usepackage{acronym}
% \usepackage{float}
\usepackage[scaled=.92]{helvet}
\usepackage{courier}
\usepackage[table]{xcolor}
\usepackage[
    a4paper,
    margin=2.54cm,
    marginparwidth=2.0cm,
    right=2.0cm,
    footskip=1.0cm
    ]{geometry}

\usepackage[
    url=true,
    style=numeric,
    alldates=long,
    dateabbrev=false
]{biblatex}
% \usepackage[
%     style=alphabetic,
%     backend=biber,
%     %backref=true
%     ]{biblatex}
\addbibresource{literature.bib}

\urlstyle{rm}

\clubpenalty=10000

\widowpenalty=10000
\displaywidowpenalty=10000
\usepackage{float}
\floatstyle{plaintop}
\restylefloat{table}
\usepackage{ifdraft}
\pagestyle{plain}
\deffootnote{1em}{1em}{\thefootnotemark.\ }

\titlehead{
    \begin{center}
        % \includegraphics[width=4cm]{images/uhh_logo_minimal.png}\\
        \includegraphics[width=8cm]{images/UHH-Logo_2010_Farbe_RGB_hires_nomargin.png}\\
        \medskip
        Department of Computer Science
    \end{center}
}

\setlength{\belowcaptionskip}{8pt plus 1pt minus 1pt}

\newcommand{\dominik}[1]{\textcolor{orange}{\textbf{Notiz: #1}}}
\newcommand{\todo}[1]{\textcolor{red}{\textbf{TODO: #1}}}
\newcommand{\cmark}{\textcolor{green}{\ding{51}}}
% \newcommand{\xmark}{\textcolor{red}{\ding{55}}}
\newcommand{\xmark}{\textcolor{red}{-}}
\newcommand{\qmark}{\textcolor{orange}{\textbf{?}}}

\title{
    Implementation of browser fingerprinting detection \\
    and integration into PrivacyScore
}

\author{Tronje Krabbe}

\date{\today}

\begin{document}
\hypersetup{hidelinks}

% \maketitle

% \newpage
% \thispagestyle{empty}
% % \addcontentsline{toc}{chapter}{Muster des Deckblatts}
% \begin{titlepage}% {{{
% \includegraphics[width=6.8cm]{images/up-uhh-logo-u-2010-u-farbe-u-rgb.pdf}
% \begin{center}\large
% 	% Universität Hamburg \par
% 	% Fachbereich Informatik
%     \vfill
% 	\makeatletter
% 	{\Large\textsf{\textbf{\@title}}\par}
% 	\makeatother
% 	\vfill
%     A thesis submitted for the degree of \\
%     \textit{Bachelor of Science} \\
%     in \\
%     \textit{Computer Science}
% 	\vfill
%     by
% 	\par\bigskip
% 	\makeatletter
% 	{\@author} \par
% 	\makeatother
% 	born 11. August 1994 in Hamburg \par
% 	matriculation number 6435002
% 	\vfill
% 	\makeatletter
% 	submitted {\@date}
% 	\makeatother
% 	\vfill
% 	Supervisor: Dipl.-Inf. Tobias Müller \par
% 	First reviewer: Prof. Dr.-Ing. Hannes Federrath \par
% 	Second reviewer: Prof. Dr. Dominik Herrmann
% \end{center}
% \end{titlepage}% }}}

\newpage
\thispagestyle{empty}
% \addcontentsline{toc}{chapter}{Muster des Deckblatts}
\begin{titlepage}% {{{
\includegraphics[width=6.8cm]{images/up-uhh-logo-u-2010-u-farbe-u-rgb.pdf}
\begin{center}\Large
	% Universität Hamburg \par
	% Fachbereich Informatik
	\vfill
	Bachelorarbeit
	\vfill
	\makeatletter
	{\Large\textsf{\textbf{\@title}}\par}
	\makeatother
	\vfill
	vorgelegt von
	\par\bigskip
	\makeatletter
	{\@author} \par
	\makeatother
	geb. am 11. August 1994 in Hamburg \par
	Matrikelnummer 6435002 \par
	Studiengang Informatik
	\vfill
	\makeatletter
	eingereicht am {\@date}
	\makeatother
	\vfill
	Betreuer: Dipl.-Inf. Tobias Müller \par
	Erstgutachter: Prof. Dr.-Ing. Hannes Federrath \par
	Zweitgutachter: Prof. Dr. Dominik Herrmann
\end{center}
\end{titlepage}% }}}


\setcounter{tocdepth}{1}
\tableofcontents

%%%%%%%%%%%%%%%%%%%%%%%%%%%%%%%%%%%%%%%%%%%%%%%%%%%%%%%%%%%%%%%%%%%%%%%%%%%%%%
%% Chapter Start                                                            %%
%%%%%%%%%%%%%%%%%%%%%%%%%%%%%%%%%%%%%%%%%%%%%%%%%%%%%%%%%%%%%%%%%%%%%%%%%%%%%%
\chapter{Introduction}
\label{chap:introduction}

When browsing the web, users can be tracked through the use of cookies, which store information on the user's computer
that can also be accessed by a remote server.
If users wish not to be trackable, they can modify or delete any cookie as they see fit.
There are, however, other methods of uniquely identifying users and tracking them during their use of the internet,
which are not as easily mitigated as cookies \cite{am_i_unique}. One of these methods is called ``browser fingerprinting'',
and it will be the focus of this thesis.

Browser fingerprinting, also known as device fingerprinting, and in the following often simply referred to as ``fingerprinting'',
works by analyzing a web browser's configuration and settings; mostly installed fonts and HTML5 canvas behavior
\cite{DBLP:conf/ccs/EnglehardtN16}, as well as
language settings, time zone settings, installed add-ons, and more. Flash is also sometimes used, though
its end-of-life lies in the foreseeable future, reportedly in the year
2020 \cite{flash_eol}.
Fingerprinting through Flash and the detection thereof will thus not be within the scope of this thesis.

The above attributes are readily available to be collected through JavaScript functions. Simply recording
all function calls made by the JavaScript front end of a website can reveal whether fingerprinting is likely to
be taking place or not \cite{faiz2014browser, panopticlick}.

Techniques used to identify and track users across different websites without
their knowledge or their ability to easily intervene, violates their privacy; by creating a fingerprint
of a user's browser, with sufficiently sophisticated techniques, one can uniquely identify one user
among hundreds of thousands \cite{am_i_unique}, re-instate any cookies the user may have deleted,
or simply track and analyze their use of any number of web services for a multitude of malicious reasons \cite{eckersley2010unique}.
Methods exist to mitigate the effectiveness of browser fingerprinting, such as the one presented
in \cite{laperdrix2015mitigating};
these are, however, usually much more complicated and specialized for the average user of a web-browser.
The author has therefore implemented a technique to recognize when a website is deploying browser fingerprinting,
and along the way explored the techniques used to do so. The resulting software has been integrated into
\url{https://privacyscore.org}, a web-service to test and rank websites according to the extent to which they
respect their users' privacy; this creates a different kind of defense against the privacy violating
methods that form the topic of this thesis: informing users about websites they use and their treatment of
sensitive information.

\section{Motivation}
There is a fair amount of related work on the topic. However, since it exists in the fields of privacy and security,
it always deals with some form of `adversary', meaning something or someone that can change and adapt to new
research and methods. It is therefore important to keep up with any changes, which can happen very quickly
in the context of software.
In this case, someone can potentially come up with some new, not-yet-detectable fingerprinting mechanism,
requiring new research to be done again, creating motivation from a technical perspective.

Tracking a user does not simply mean recognizing whether a visitor to one's website is a new or an existing user.
Identifying information can be passed on or sold to partners, advertisers, or even governments, in order to
construct a rich browsing history of a user. Users are required by EU law to be informed of cookies being set
by websites CITATION NEEDED, yet fingerprinting can and does happen quietly.
The work of this thesis aims to create transparency for users in regards to
the websites they visit, giving a social motivational aspect.

\section{Fundamentals}
The following section will explain some fundamental knowledge that is
referenced throughout this thesis.
\dominik{ist diese section cool oder soll sie weg?}

\subsection{Fingerprint}
\label{fundamentals:fingerprint}
Throughout the thesis, terms such as ``fingerprint'', ``browser fingerprint'', etc.
will be used largely interchangeably.
A fingerprint, in most contexts, is something that can uniquely identify something else.
The obvious example is a human's fingerprint, which is unique to each finger of each human,
with very few exceptions, if any.\footnote{\url{https://www.interpol.int/INTERPOL-expertise/Forensics/Fingerprints}}

\subsection{Entropy}
\label{fundamentals:entropy}
Entropy is the most important aspect of a fingerprint.
Peter Eckersley describes entropy as
``a mathematical quantity which allows us to measure how close a fact comes to revealing somebody's identity uniquely''.\footnote{\url{https://www.eff.org/deeplinks/2010/01/primer-information-theory-and-privacy}}
Entropy is often measured in bits, because it measures the different
possible values something can take.
To give an example, the outcome of a coin toss has one bit of entropy,
as it can have two different values: heads or tails.
A random, unknown human's identity contains about 33 bits of entropy,
as two to the power of 33 is 8 billion, and there are some 7.6 billion humans
alive at the time of writing.
If a browser fingerprint can provide 33 bits of information, it can be used
to uniquely identify any person, so long as all fingerprints are unique.

\subsection{JavaScript}
\label{fundamentals:javascript}
JavaScript\footnote{\url{https://developer.mozilla.org/en-US/docs/Web/JavaScript}} is,
at the time of writing, the primary programming languages understood
by modern web browsers. While HTML and CSS dictate the visual layout and style
of a web page, JavaScript provides its functionality.
JavaScript is important in the context of this thesis, because
it can be used to construct a browser fingerprint. JavaScript can query
a browser's or device's attributes, transform them, and transmit them
to a third party.
Within this thesis, the JavaScript code executed by websites
when visited, will be saved and analyzed.
This constitutes the primary source of data for the thesis.

\subsubsection{Obfuscation}
\label{fundamentals:obfuscation}
JavaScript source code can be obfuscated. Obfuscation means to change the code
until it is no longer possible for a human to read the code, at least not within
a reasonable time frame. We have encountered many instances of obfuscated JavaScript
code. For an example of JavaScript obfuscation, consider the code samples in
Figure~\ref{code:javascript_hello_world} and Figure~\ref{code:javascript_obfuscated}.
They are functionally equivalent. The code in Figure~\ref{code:javascript_obfuscated}
has been obfuscated by JavaScript Obfuscator\footnote{\url{https://javascriptobfuscator.com}}.
A more extreme example for obfuscation of JavaScript code is provided by JSFuck\footnote{\url{jsfuck.com}},
the output of which cotains only six distinct characters, while still being technically valid JavaScript.

\begin{figure}
\centering
\begin{minted}{javascript}
alert('Hello, world!');
\end{minted}
\caption{A JavaScript ``Hello, World'' program}
\label{code:javascript_hello_world}
\end{figure}

\begin{figure}
\centering
\begin{minted}{javascript}
var _0xd533 = [
    "\x48\x65\x6C\x6C\x6F\x2C\x20\x77\x6F\x72\x6C\x64\x21"
];

alert(_0xd533[0]);
\end{minted}
\caption{An obfuscated JavaScript ``Hello, World'' program}
\label{code:javascript_obfuscated}
\end{figure}

\subsection{Canvas}
\label{fundamentals:canvas}
The canvas \cite{w3ccanvas} is a HTML5 element, upon which a web page can draw,
using one of several methods. For instance, WebGL allows rendering of complex
3D graphics on a canvas.
The canvas can also be manipulated by JavaScript. For this thesis,
the canvas plays an important role. It enables the generation
of images with high entropy values in regards to their fingerprinting value.
We will show that many websites draw on a canvas, extract the resultant raw pixel
values from the canvas, and use it to contribute to the generation of a fingerprint.
This is possible for a number of reasons. For one, a JavaScript program
can instruct the browser it is running in to draw emoji on a canvas.
If the browser supports emoji, they will be drawn in a certain way, usually dictated
by the device the browser is running on. Apple's emoji look different from Google's
emoji, for instance.\footnote{\url{https://emojipedia.org/apple/}}\footnote{\url{https://emojipedia.org/google/}}
If emoji are not supported, fallback behavior often means that black squares will
be drawn in their place.
Thus, drawing simple emoji on a canvas can give up information about a device's vendor.
A similar approach can be taken with fonts.
Creating a WebGL rendering context and rendering images on it can reveal
information about the GPU and associated driver software of the host device.

\todo{noch mehr?}

\section{Goal and Leading Questions}
The overall goal of this thesis is to create a software that can reliably report whether a website carries out
fingerprinting, and to integrate it into \url{https://privacyscore.org}. To achieve this goal,
we must inspect the state of browser fingerprinting across the web, and gain new insights into the topic.
To this end, we will attempt to answer the following leading questions.

\textit{What are the most commonly used fingerprinting techniques and how can they be identified by our software?}\\
Identifying the most common techniques is important, as it will allow the detection of the broadest spectrum
of fingerprinters. What are they, and how can we best identify them?

\textit{How can false positives and false negatives be minimized?}\\
False positives are, in our case, websites which we identify as fingerprinters, but which do not actually
perform fingerprinting. False negatives, conversely, are sites that do construct fingerprints, but which we
label as non-fingerprinters.
Both are detrimental to the results of the software and thus must be minimized.

\textit{How do multiple different fingerprinting techniques relate?}\\
We assume that a fingerprinter is more likely to use more than one fingerprinting technique,
and that the majority of fingerprinters also use the highest amount of different techniques.
Is this the case? Which techniques are most often used together? Why?


% \subsection{What are the most commonly used fingerprinting methodologies?}
% Methods to fingerprint change. Flash, for example, can be used to construct a fingerprint, but is expected
% to be used less and less, since its end of life has been announced.
% So, what are the most-used methods \textit{today}, and how much will these change over time? Which methods
% can be expected to still be used in a year, and which will be deprecated soon?
%
% \subsection{What are the most commonly used fingerprinting libraries and how can they be found?}
% Perhaps the most used library is a closed-source, proprietary product by some large conglomerate,
% with no publicly available information. If its use can be reliably recognized, many websites could be found guilty
% almost immediately. This, however, isn't a given, and it may take considerable effort to answer this question.
%
% \subsection{How can false positives be minimized?}
% False positives, i.e. websites that are identified as fingerprinters, but don't actually employ any such technology,
% are detrimental to the overall results of the software. How can they be minimized or, ideally, eliminated?
%
% \subsection{What does a good/fair metric look like?}
% The impact of false positives can be diminished by employing a rating system or metric to rate a website's
% ``fingerprinting score''. Instead of reporting boolean results (true or false), the software should
% report a number or score that represents the likelihood of fingerprinting taking place, as determined by the software.
% How is such a score calculated, and which attributes are used in this calculation?
%
% \subsection{Can all this be achieved with acceptable performance?}
% This is as much a leading question as it is a goal: the software \textit{must} exhibit adequate performance.
% Since a key component of the stated goal is to integrate the software into \url{https://privacyscore.org},
% it must be able to analyze a website within seconds, to allow acceptable performance of PrivacyScore itself.
% If the software were designed to simply be run once for some large number of websites, so that the results
% could be displayed, it would be fine if it took some arbitrary amount of time. But in order to be useful
% in the desired context, it is imperative that performance is at least something to be kept in mind.

\section{Organization}
The structure of this thesis is shown in the following.
In Chapter~\ref{chap:related_work}, we present related work, split into two main groups:
related work on the topic of browser fingerprinting and related work on
browser fingerprinting detection.
Chapter~\ref{chap:data_acquisition} will showcase our data acquisition
methods and discuss some basic analyses.

\todo{alle chapter erklären}

% \section{Methods and Approach}

%%%%%%%%%%%%%%%%%%%%%%%%%%%%%%%%%%%%%%%%%%%%%%%%%%%%%%%%%%%%%%%%%%%%%%%%%%%%%%
%% Chapter End                                                              %%
%%%%%%%%%%%%%%%%%%%%%%%%%%%%%%%%%%%%%%%%%%%%%%%%%%%%%%%%%%%%%%%%%%%%%%%%%%%%%%


%%%%%%%%%%%%%%%%%%%%%%%%%%%%%%%%%%%%%%%%%%%%%%%%%%%%%%%%%%%%%%%%%%%%%%%%%%%%%%
%% Chapter Start                                                            %%
%%%%%%%%%%%%%%%%%%%%%%%%%%%%%%%%%%%%%%%%%%%%%%%%%%%%%%%%%%%%%%%%%%%%%%%%%%%%%%
\chapter{Related Work}
\label{chap:related_work}
This Chapter will give an overview of previous works. It is split into two sections.
The first section presents existing work on the topic of fingerprinting. The second one will show existing work on
the detection of fingerprinting.

\section{Browser Fingerprinting}
There are closed-source implementations of browser/device fingerprinting. These are unavailable to us.
However, we can turn to several open-source projects which implement fingerprinting for deeper insights into this technology,
and to learn how we may be able to identify these closed implementations.

Essentially, fingerprinting works by combining data available to front-end JavaScript code into a string,
usually by hashing some data structures. This string is the fingerprint.
Different fingerprinting methods can be distinguished by the different attributes they use to construct their fingerprint.

For instance, one of the attributes providing the most entropy is acquired by so-called canvas fingerprinting, which works by drawing
several symbols like geometric shapes, emoji, and so on, on an HTML5 canvas element \cite{laperdrix2016beauty}.
The script in Appendix~\ref{app:canvas_fingerprinting_script} \todo{fingerprinting script} shows an example of canvas fingerprinting.
Usually, letters, words, or whole sentences are also written onto the canvas, oftentimes with multiple different fonts.
The resulting image can be serialized into a representation that can be included in the aforementioned fingerprint string.
The fingerprint contains information about the host OS, and the hardware on which it runs.

For a concrete example, \textit{Am I Unique?} writes the same pangram\footnote{A sentence containing every letter of the alphabet.}
twice onto the canvas, using two different fonts, and each time followed by a special Unicode character, an emoji.
Finally, a rectangle in a specific color is drawn \cite{laperdrix2016beauty}.

So, while different fingerprinting scripts are all likely to use canvas fingerprinting, they can differ in the exact
method they use to create the canvas-related aspect of the fingerprint.

The methods described in this section can shed some light onto the different attributes used, and their computation,
and how much entropy each can contribute to a fingerprint.

We have compiled an overview of which attributes are used by which implementation in Table~\ref{app:attribute_table}.


\subsection{Am I Unique?}
\label{related_work:am_i_unique}
\textit{Am I Unique?} \cite{laperdrix2016beauty} is a web service which creates
a fingerprint with the press of a button.\footnote{\url{https://amiunique.org}}
Its purpose is to educate users about fingerprinting
\textit{Am I Unique?} borrows part of its JavaScript fingerprinting code from
Fingerprintjs2, which is discussed below.\footnote{\url{https://github.com/DIVERSIFY-project/amiunique/blob/master/website/public/javascripts/webGL.js\#L2}}


% FIXME there's a new version of Panopticlick that was released after the paper. Does it include canvas fingerprinting?
\subsection{Panopticlick}
\label{related_work:panopticlick}
The EFF's Panopticlick project is similar to \textit{Am I Unique?}, in that it aims to learn and inform about
fingerprinting. It also offers a way to test any tracking protection a user may have enabled by simulating
tracking domains \cite{panopticlick}. The attributes used by Panopticlick to construct a fingerprint differ slightly from those used by
\textit{Am I Unique?}, though overall, they are similar.

Eckersley showed that the list of installed fonts and the list of installed plugins are the most identifying properties
\cite{eckersley2010unique}. This finding, however, did not yet take canvas fingerprinting into consideration.


\subsection{Fingerprintjs2}
\label{related_work:fingerprintjs2}
Fingerprintjs2 \cite{fingerprintjs2} is a popular JavaScript program that
can be used to easily construct a fingerprint.\footnote{4891 stars and 712 forks on GitHub. Visited on February 4, 2018}
It can use a larger variety of attributes when compared to the two previously mentioned services by default.
Which attributes are used to construct the fingerprint can be configured.

A crawl from January 27, 2018, of 4929 sites shows that 336 sites use Fingerprintjs2, for a percentage of about 6.82\%.
This number was found by simply querying the database for function names that are consistent with
those used in Fingerprintjs2, and therefore do not include sites which use an obfuscated Fingerprintjs2.
Fingerprintjs2 has a few functions starting with the prefix \textit{getHasLied}, like \textit{getHasLiedOs}, the purpose
of which is to report whether a browser tampered with certain information, for instance in the interest of mitigating
fingerprinting.\footnote{\url{https://github.com/Valve/fingerprintjs2/wiki/Browser-tampering} Visited on February 8, 2018}
Due to the peculiar name choice, and the fact that the same sites that call one \textit{getHasLied}
function also call all others, we can be quite certain that they all include Fingerprintjs2.


\subsection{Attribute Overview}
Table~\ref{table:attributes_used} shows which of the presented fingerprinting services/libraries use which attributes of the
browser \cite{am_i_unique,panopticlick,fingerprintjs2}.

\renewcommand{\arraystretch}{1.2}
\begin{table}
\centering
\caption{Different Browser Attributes used by Fingerprinting Projects}
\begin{tabular}{ l c c c }
    \toprule
    Attribute & Am I Unique? & Panopticlick & Fingerprintjs2 \\
    \midrule
    User Agent & \cmark & \cmark & \cmark \\[3px]
    Accept Header & \cmark & \cmark & \xmark \\[3px]
    Content Encoding & \cmark & \xmark & \xmark \\[3px]
    Content Language & \cmark & \xmark & \xmark \\[3px]
    List of plugins & \cmark & \cmark & \cmark \\[3px]
    Cookies enabled? & \cmark & \cmark & \xmark \\[3px]
    Local storage available? & \cmark & \xmark & \cmark \\[3px]
    Session storage available? & \cmark & \xmark & \cmark \\[3px]
    Timezone & \cmark & \cmark & \cmark \\[3px]
    Screen resolution & \cmark & \cmark & \cmark \\[3px]
    Screen color depth & \cmark & \cmark & \cmark \\[3px]
    List of fonts & \cmark & \cmark & \cmark \\[3px]
    List of HTTP headers & \cmark & \xmark & \xmark \\[3px]
    Platform & \cmark & \cmark & \cmark \\[3px]
    Do Not Track Header present? & \cmark & \cmark & \cmark \\[3px]
    Canvas & \cmark & \cmark & \cmark \\[3px]
    WebGL & \cmark & \cmark & \cmark \\[3px]
    WebGL Vendor & \cmark & \xmark & \cmark \\[3px]
    WebGL Renderer & \cmark & \cmark & \cmark \\[3px]
    Use of an ad blocker & \cmark & \xmark & \cmark \\[3px]
    JavaScript allowed? & \xmark & \cmark & \xmark \\[3px]
    System Language & \xmark & \cmark & \cmark \\[3px]
    Touchscreen support & \xmark & \cmark & \cmark \\[3px]
    IndexedDB available? & \xmark & \xmark & \cmark \\[3px]
    Open DB? & \xmark & \xmark & \cmark \\[3px]
    CPU class & \xmark & \xmark & \cmark \\[3px]
    Available processors & \xmark & \xmark & \cmark \\[3px]
    Device memory & \xmark & \xmark & \cmark \\[3px]
    \bottomrule
\end{tabular}
\label{table:attributes_used}
\end{table}



\subsection{Fingerprint Central}
\label{related_work:fp_central}
Fingerprint Central, or FP Central, is a website which ``aims at studying the diversity
of browser fingerprints and providing developers with data to help them design good defenses''.\footnote{\url{https://fpcentral.tbb.torproject.org/}}
Much like Panopticlick or \textit{Am I Unique?}, one can have the site generate a device fingerprint,
and view the exact components of the fingerprint, as well as statistics about other users' fingerprints.
Of note is the fact that FP Central does not use canvas fingerprinting at this time.
However, the service is still in its beta phase, so this may change in the future.
Another interesting aspect of FP Central is the math fingerprinting.
High-precision mathematical functions can have operating system specific behavior, and thus provide
entropy.\footnote{\url{https://trac.torproject.org/projects/tor/ticket/13018}}
FP Central also performs audio context fingerprinting, which is also not present in any of the previously presented
fingerprinters.


\section{Browser Fingerprinting Detection}
At a basic level, in order to detect fingerprinting, all JavaScript calls to the relevant attributes,
like User Agent, HTML5CanvasElement, and so on, need to be recorded. Then, a decision can be made about whether the number
of calls is suspicious, or constitutes normal website behavior.

However, since these attributes also have legitimate uses, this is not a trivial task.
This section will present previous work on this topic.


\subsection{FPDetective}
\label{related_work:fpdetective}
Acar et al. provide some useful metrics to differentiate between fingerprinting and legitimate use of
browser attributes \cite{DBLP:conf/ccs/AcarJNDGPP13}. They have implemented a fingerprinting-detection
framework called \textit{FPDetective}.
\textit{FPDetective} focuses on font-based fingerprinting, and uses
\textit{PhantomJS} to log JavaScript function calls and \textit{Chromium} to log Flash actions.
Much like OpenWPM, their crawler logs accesses to certain browser and device properties, such as
those contained within the \textit{window.navigator} object.
Popular attributes like the HTML5 Canvas and WebGL were ignored in the study and by the framework's logging.

Partly automated, their process consisted of an automated analysis in regards to font fingerprinting,
and a manual analysis of JavaScript and decompiled Flash code.
Their focus on fonts derives from the assumption that a website that
enumerates fonts is a likely fingerprinter.

\textit{FPDetective} classifies a script as a fingerprinter if ``it loads more than 30 system fonts, enumerates plugins
or mimeTypes, detects screen and navigator properties, and sends the collected data back to a remote server''
\cite{DBLP:conf/ccs/AcarJNDGPP13}.
In Chapter~\ref{todo}, we show that this metric is \todo{accurate/useful or not?}.

As Flash is nearing its end-of-life, this thesis opts not to analyze Flash programs
delivered by websites, which means \textit{FPDetective}'s extensive work on Flash-based fingerprinting
is, of little use for us.


\subsection{OpenWPM}
\label{section:openwpm}
Englehardt et al. have developed a framework called \textit{OpenWPM}, which can be used to analyze the way a website
handles users' privacy. They have used this to analyze the tracking behavior of one million websites
\cite{DBLP:conf/ccs/EnglehardtN16,englehardt2016census}.

Englehardt et al. differentiate between distinct forms of fingerprinting.
They classify a script as a canvas-fingerprinter, if a there is a canvas larger than 16 by 16 pixels, and text is
written to it in either at least two different colors, or with 10 distinct characters.
A canvas-fingerprinter script should not call the \textit{save}, \textit{restore} or \textit{addEventListener}
methods of the rendering context. Finally, the script has to call either \textit{toDataURL}
or \textit{getImageData} for an area of at least 16 pixels in width and height.
They have manually verified the accuracy of their heuristic, and found only four false positives
out of 3493 scripts.
We will show that this heuristic is \todo{good/not good?}.

Englehardt et al. make a distinction between canvas fingerprinting and canvas font fingerprinting.
They classify a script as a canvas font fingerprinter if it sets the \textit{font} property
of a canvas at least 50 times, and also calls the \textit{measureText} method at least 50 times.
They have manually verified that there were no false positives for this metric.

Their findings show that only about 1.6\% of the analyzed sites
employ canvas-based fingerprinting. However, they also show that among the top 1,000 sites,
5.1\% of sites employ canvas-based fingerprinting.

Their work also examines new, previously unknown methods of fingerprinting, such as canvas font fingerprinting, which
works by rendering text in a large number of different fonts on a canvas element; WebRTC-based fingerprinting, which,
as the name suggests, abuses the WebRTC framework; as well as AudioContext fingerprinting, and Battery API fingerprinting.

OpenWPM uses Firefox, Selenium, and custom JavaScript injected into loaded sites to log calls to relevant browser attributes.
It constitutes the technological basis of this thesis' work, as it allows easy crawling of websites, while recording
most if not all relevant information.


\subsection{Tor Browser}
\label{related_work:tor_browser}
\textit{The Design and Implementation of the Tor Browser [DRAFT]}
states that ``[the authors] believe that the HTML5 Canvas is the single largest fingerprinting threat browsers face today'',
as a canvas offers easy and high-entropy fingerprinting
\cite{acar2014web,mowery2012pixel}.\footnote{\url{https://www.torproject.org/projects/torbrowser/design/} Visited on February 8, 2018}
The Tor Browser can alert users to data collection from a HTML5 Canvas element.
Data collection, in this context, means using a function like \textit{toDataURL} or \textit{measureText} on the element.

%%%%%%%%%%%%%%%%%%%%%%%%%%%%%%%%%%%%%%%%%%%%%%%%%%%%%%%%%%%%%%%%%%%%%%%%%%%%%%
%% Chapter End                                                              %%
%%%%%%%%%%%%%%%%%%%%%%%%%%%%%%%%%%%%%%%%%%%%%%%%%%%%%%%%%%%%%%%%%%%%%%%%%%%%%%


%%%%%%%%%%%%%%%%%%%%%%%%%%%%%%%%%%%%%%%%%%%%%%%%%%%%%%%%%%%%%%%%%%%%%%%%%%%%%%
%% Chapter Start                                                            %%
%%%%%%%%%%%%%%%%%%%%%%%%%%%%%%%%%%%%%%%%%%%%%%%%%%%%%%%%%%%%%%%%%%%%%%%%%%%%%%
\chapter{Data Acquisition and Analysis}
\label{chap:data_acquisition}
This Chapter presents the data set used to conduct the experiments used in this thesis.
We will describe how the data was acquired, as well as some basic properties of the set.

\section{Acquiring Site Information}
As described in Section~\ref{section:openwpm}, we have used OpenWPM\footnote{\url{https://github.com/citp/OpenWPM}} to gather data about
website behavior.
The parameters used with OpenWPM to gather the data can be found in Appendix~\ref{app:params}.

OpenWPM saves most data in SQLite3 databases.\footnote{\url{https://www.sqlite.org/index.html}}
This includes the data interesting to us: JavaScript calls, function names, and script URLs.
Python3's standard library makes it possible to interact with SQLite3
databases.\footnote{\url{https://docs.python.org/3/library/sqlite3.html}}
Thus, all of our analyses are implemented in Python3.
Physical distributions of this thesis include the source code on a CD-ROM.
We will show various methods used in our analyses in Chapter \todo{Kapitel schreiben und dann hier ref'en}

Alexa regularly compiles a list of the top one million websites, ordered by the number of
visitors.\footnote{\url{https://www.alexa.com/topsites}}
We have analyzed the top 5000 websites, as described by a version of this list acquired from
\url{http://s3.amazonaws.com/alexa-static/top-1m.csv.zip} on December 19, 2017.

Multiple data sets based on this list were gathered. Due to timeouts or other
errors, not all 5000 sites were actually crawled each time.
The primary data set we used was crawled on January 27, 2018, and includes 4929 sites with distinct URLs. Some of these sites
may be duplicates of each other. For instance, the URL \url{https://fbcdn.net/} for Facebook's content delivery
network simply points to \url{https://facebook.com/}, both of which are included in the data set.

We make the distinction between the primary data set and secondary data sets.
We have run all analyses on the primary data set, and we will show
that the noted behavior is consistent by repeating some analyses
on the secondary sets.
We do this because running all analyses on all sets is beyond the time
constraints of this work.
In the following, we will refer to the primary data set as simply ``the data set''.

\section{Data Properties}
\label{section:data_properties}
In this section, we will give an overview of basic properties of the primary data set,
with special regard to their relevance for fingerprinting.
Table~\ref{table:dataprops} shows some of these properties.

\renewcommand{\arraystretch}{1.2}
\begin{table}
\centering
\caption{Analyzed Website's Behavior}
\begin{tabular}{l r r}
    \toprule
% \arrayrulecolor{hlinegray}
    & amount & percentage \\
    All sites & 4929 & 100\% \\
    \midrule
    \textit{Basic attribute look ups} & & \\
    \textit{userAgent} & 4424 & 90\% \\ %89.8
    \textit{language} or \textit{languages} & 3985 & 81\% \\ %80.8
    \textit{colorDepth} & 3931 & 80\% \\ %79.8
    \textit{localStorage} & 3772 & 77\% \\
    \textit{platform} & 3275 & 66\% \\ %66.4
    \textit{sessionStorage} & 2860 & 58\% \\
    \textit{cookieEnabled} & 2823 & 57\% \\ %57.3
    \textit{doNotTrack} & 1406 & 29\% \\ %28.5
    \textit{oscpu} & 580 & 12\% \\ %11.8
    \midrule
    \textit{Using multiple of above attributes} & & \\
    at least 2 & 4338 & 88\% \\
    at least 3 & 4245 & 86\% \\
    at least 4 & 3968 & 81\% \\
    at least 5 & 3572 & 72\% \\
    at least 6 & 2906 & 59\% \\
    at least 7 & 1963 & 40\% \\
    at least 8 & 1138 & 23\% \\
    9 attributes & 473 & 10\% \\
    \midrule
    \textit{Canvas-related functions or attributes} & & \\
    Using any Canvas-related JavaScript & 1657 & 34\% \\ %33.6
    \textit{toDataURL} or \textit{getImageData} & 884 & 18\% \\ %17.9
    \textit{toDataURL} & 768 & 16\% \\ %15.6
    \textit{getImageData} & 252 & 5\% \\ %5.1
    \midrule
    Calling functions defined by \textit{Fingerprintjs2} & 336 & 7\% \\ %6.8
    \bottomrule
\end{tabular}
\label{table:dataprops}
\end{table}

The first block of Table~\ref{table:dataprops}, titled \textit{Basic attribute look ups},
shows how many sites looked up certain browser or device attributes.
We label these as ``basic'', because a JavaScript program can access them with a single function
call. In this section of the table, we can see that, for example, 90\% of sites in the set access the
\textit{userAgent} property.
How many websites access which attribute is interesting to us because of the basic attributes' ease of access.
As their values are easy to acquire, one might expect that these attributes are consistently used by different fingerprinters.

The next section in the table shows how many sites use more than one of the basic attributes.
Note that the rows are not mutually exclusive. This means that, e.g. the 4245 sites using ``at least'' three
attributes also contain all sites that use more than three attributes.
This information is useful, because fingerprints are and need to be constructed from
multiple data points. If they weren't, they would not be unique enough \cite{panopticlick}.

The final block of Table~\ref{table:dataprops} is focused on the HTML5 Canvas element.
More specifically, we show how many sites use any kind of Canvas-related JavaScript,
as well as how many use the specific data extraction methods.
A website qualifies as using ``any Canvas-related JavaScript'' when it makes use
of any function calls that are part of the Canvas API.\footnote{\url{https://developer.mozilla.org/en-US/docs/Web/API/Canvas_API}}
The two methods that can be used for data extraction are
\textit{toDataURL}\footnote{\url{https://developer.mozilla.org/en-US/docs/Web/API/HTMLCanvasElement/toDataURL}}
and
\textit{getImageData}\footnote{\url{{https://developer.mozilla.org/en-US/docs/Web/API/CanvasRenderingContext2D/getImageData}}}.
This part of the table is interesting to us, as the Canvas element is one of the
most useful elements for the construction of a fingerprint, as described
in Section~\ref{fundamentals:canvas}.

Of the 884 sites that use \textit{toDataURL} or \textit{getImageData} on a HTML5 Canvas object, which suggests
a high likelihood of canvas fingerprinting, 536, or about 61\%, also use eight or more of the basic attributes.
These 536 sites (not shown in the table) are good candidates for likely fingerprinters.
And indeed, 147 of them are contained in the 336 sites that we have identified as users of \textit{Fingerprintjs2}.
\textit{Fingerprintjs2}, as discussed in Section~\ref{related_work:fingerprintjs2},
can be configured to include any number of attributes
in its generated fingerprint. From the above, we can also conclude that 189 users of \textit{Fingerprintjs2}
do not use either of the data extraction methods of the Canvas element. This is curious because
of the Canvas' high entropy potential.
In Section \todo{}, we will show that this is \todo{not?} an anomaly in the data we have collected.


\section{\textit{toDataURL} Usage}
\label{sec:todataurl}
In Section~\ref{section:data_properties} we suggested that a use of \textit{toDataURL}
suggests a high likelihood of canvas fingerprinting.
To show the plausibility of this claim, we have performed a manual analysis of the JavaScript served by
75 sites that use \textit{toDataURL} by hand.
These 75 sites were randomly sampled from the list of sites which call \textit{toDataURL},
but have not been identified as users of FingerprintJS2.

\begin{figure}
\centering
\begin{minted}{sql}
SELECT DISTINCT script_url
FROM javascript
WHERE symbol = 'HTMLCanvasElement.toDataURL';
\end{minted}
\caption{SQL query to select all JavaScript programs using \textit{toDataURL}.}
\label{code:todataurl_users_query}
\end{figure}

OpenWPM gives us a SQLite database, which includes a \textit{javascript} table.
This table includes a column called \textit{script\_url}, which contains
the URL to a JavaScript file, and a column called \textit{symbol}, containing
the name of the symbol accessed by a JavaScript program.
Using the query depicted in Figure~\ref{code:todataurl_users_query}, we can find all scripts
that access the symbol \textit{HTMLCanvasElement.toDataURL}.

The analysis of these scripts was done by looking for tell-tale signs of fingerprinting
in the often minified JavaScript code.
These signs are enumeration of fonts, writing pangrams to canvases, writing nonsense
to canvases, never displaying canvases, enumeration of plugins, and suspicious function
names like \textit{getFingerprint}.

Following the analysis, we claim with confidence that 51 out of the 75 sites are fingerprinters.
Table~\ref{table:todataurl_users} shows more details of this analysis.

\begin{table}
\centering
\caption{\textit{toDataURL} User's Behavior}
\begin{tabular}{l r r}
    \toprule
    & amount & percentage \\
    \midrule
    \textbf{Analyzed sites} & 75 & 100\% \\
    \midrule
    \textbf{Fingerprinting sites} & 51 & 67\% \\
    Using a popular third-party script & 26 & 34\% \\
    Using functions with suspicious name & 11 & 15\% \\
    Using a popular pangram on canvas & 4 & 5\% \\
    \midrule
    \textbf{Inconclusive} & 16 & 21\% \\
    \midrule
    \textbf{Legitimate users} & 9 & 12\% \\
    Emoji-detection & 7 & 9\% \\
    Applying blur to image & 1 & 1\% \\
    Lazily loading an image & 1 & 1\% \\
    \bottomrule
\end{tabular}
\label{table:todataurl_users}
\end{table}

The first block of Table~\ref{table:todataurl_users} shows the sites
we have deemed to be fingerprinters. Of these, 26 use one of several scripts
that are served by many of the sites in the dataset. We will present these
in more detail in \todo{Section}. This is interesting, because
the behavior of widely used scripts can be used to tune our scoring system.
However, when assigning a score, looking for suspicious scripts based on
name or URL alone, is not practical. This is because a fingerprinter
can hide this from us easily, if they wanted to.
One way to hide a script is to obfuscate it instead of simply minifying.
Another way to hide its origin is to serve it as a first-party script,
not a third-party script, and replacing its name in a URL with random
letters or words.

The block showing the fingerprinting sites also show that 11 of them
use a suspicious name for a function. To give some examples:
\textit{getCanvasFp}, \textit{getCanvasPrint}, \textit{canvasFingerprint}.
We assume that any function would not be named in such a way if it did not
indeed perform fingerprinting.
The scripts using the suspicious names additionally showed other behavior
exhibiting the signs of fingerprinting.
Looking for suspicious function names while scoring should be avoided,
as adversaries can simply change the function names to avoid or
limit detection, as many scripts already do through obfuscation.

% Cwm fjordbank glyphs vext quiz, 😃
\begin{figure}
\centering
\includegraphics[scale=0.4]{images/fjordbank.png}
\caption{Pangram with emoji as used by \textit{Am I Unique?}, among others}
\label{code:pangram}
\end{figure}

Four of the found fingerprinters used the popular pangram shown in
Figure~\ref{code:pangram} to perform canvas fingerprinting.
Unfortunately, this is not useful in scoring future fingerprinters,
as they can easily change the pangram they use.

We label sixteen of the \textit{toDataURL} users as inconclusive.
Their JavaScript was too obfuscated to draw meaningful conclusions.
While obfuscation may itself be suspicious, many sites could perform
it simply to keep their codebase closed-source.

The final block of Table~\ref{table:todataurl_users} shows sites which
made ``legitimate'' use of \textit{toDataURL}. Seven sites used a canvas
to detect how well emoji were supported. This is done by drawing emoji
to a canvas, and analysing the array returned from \textit{toDataURL}.
If emoji are not supported by the system, they will be drawn as black
squares or other non-emoji symbols.

One site used \textit{toDataURL} seemingly to apply blur to an image.
The final entry in Table~\ref{table:todataurl_users} shows one site
using a JavaScript module or library seemingly called ``lazyimage''
to lazily load an image. This means to include a placeholder
image on the site, and only load the actual image when needed.


\section{Audio Fingerprinting}
The distinct JavaScript interfaces related to audio present in our data set,
as recorded by OpenWPM, are:
\textit{AudioContext},
\textit{OfflineAudioContext},
\textit{AnalyserNode},
\textit{GainNode} and \textit{ScriptProcessorNode}.\footnote{See also: \url{https://developer.mozilla.org/en-US/docs/Web/API/Web_Audio_API}}

\begin{figure}
\begin{minted}{sql}
SELECT DISTINCT site_visits.site_url
FROM site_visits
  JOIN javascript
    ON site_visits.visit_id = javascript.visit_id
WHERE javascript.symbol LIKE '%audio%'
   OR javascript.symbol LIKE 'AnalyserNode%'
   OR javascript.symbol LIKE 'GainNode%'
   OR javascript.symbol LIKE 'OscillatorNode%'
   OR javascript.symbol LIKE 'ScriptProcessorNode%';
\end{minted}
\caption{Query to select all users of audio-related JavaScript calls}
\label{code:audio_query}
\end{figure}

\begin{table}
\centering
\caption{Audio API User's Behavior}
\begin{tabular}{l r r}
    \toprule
    & amount & percentage \\
    \midrule
    \textbf{Analyzed sites} & 64 & 100\% \\
    \midrule
    \textbf{Fingerprinting sites} & 46 & 72\% \\
    PerimeterX script users & 15 & 23\% \\
    Facebook script users & 13 & 20\% \\
    \midrule
    \textbf{Inconclusive} & 17 & 27\% \\
    users of too obfuscated b2c script & 11 & 17\% \\
    otherwise too obfuscated & 5 & 8\% \\
    otherwise inconclusive & 1 & 2\% \\
    \midrule
    \textbf{Legitimate users} & 1 & 2\% \\
    \bottomrule
\end{tabular}
\label{table:audio_users}
\end{table}

The SQL query in Figure~\ref{code:audio_query} selects all sites that
use a JavaScript interface or symbol related to audio.
It returns 64 rows when executed on our data set, meaning there are 64 potential audio
fingerprinters, which represents about $1.2\%$ of our set.
We have performed a manual analysis of these sites in regards to audio
fingerprinting similar to the one we have performed for \textit{toDataURL}.
The results are displayed in Table~\ref{table:audio_users}.

The most important takeaway from Table~\ref{table:audio_users} could be that we were
able to find just one legitimate use of the JavaScript Audio API.
``Legitimate'' in this case means that it is used to generate and play audible sounds
for a visitor to hear.

Given the high amount of ``inconclusive'' entries, we have visited each Audio user labeled as such
in a Chromium web browser, which produces the version string ``Version 65.0.3325.181 (Developer Build) (64-bit)''.
This browser was otherwise unmodified, meaning there were no plugins, such as ad-blockers, installed.
Note that this version the the Chromium web browser supports the Audio API, which we have manually verified
by creating an instance of \textit{AudioContext} in the developer tools' console.

We have visited each site and waited for it to finish loading completely, and waited a further 10 seconds,
exactly like OpenWPM in our automated data collection.
One of these sites has played audio, although via an embedded YouTube livestream.
This, as far as we could tell, does not use the Audio API.

Thus, we are confident that our data set indeed contains only a single legitimate use of
the Audio API.

Table~\ref{table:audio_users} further shows the use of three prominent, reoccuring scripts, the most used
of which is one provided by PerimeterX\footnote{\url{https://www.perimeterx.com/}}.
PerimeterX offers ``Bot Detection and Anti Bot Protection with Unparalleled Accuracy'',
according to their website's home page. To this end, they seem to construct a browser fingerprint.
Their script accesses numerous properties of the \textit{window.navigator} object, uses \textit{toDataURL},
\textit{WebGL}, the Web Audio API, and enumerates 32 different fonts.

The second most used script originates from Facebook. It can be found under the same URL
for each of the 13 sites in our set which use it.\footnote{\url{https://cdn.atlassbx.com/FB/11122200772940/browser_features1488235112.js}, visited on March 30, 2018}
Along with the Web Audio API, it uses the Canvas API, enumerates 26 distinct fonts and accesses other
properties of the \textit{window} and \textit{window.navigator} objects which are often used in fingerprinting.
We have also found the ``mmmmmmmmmmlli'' magic string in the JavaScript code, which we will further discuss
in Section~\ref{sec:font_fingerprinting}.
Thus, we also classify it as a fingerprinter.

We refer to the final script as ``the b2c script'', although it may in fact be more than one distinct
script. We have observed it to be served from mutliple different URLs,
all beginning with \textit{api.b2c.com/api/init-}, and ending with a mix of letters and numbers and the \textit{.js} extension.
All of these scripts were much too obfuscated to make any sense of. We have thus labeled their behavior ``inconclusive''.
However, we assert that it is very likely that these scripts are indeed fingerprinters.
OpenWPM has recorded 31 distinct symbols being accessed by these ``b2c'' scripts, among them
those related to the Canvas API, Web Audio API, WebGL, \textit{localStorage}, \textit{sessionStorage},
as well as multiple ``basic properties''.

We conclude that the Audio API is a prime indicator of browser fingerprinting, as the instances of
fingerprinting use far outweigh its use of actual audio playback.


\section{Font Fingerprinting}
\label{sec:font_fingerprinting}
Font fingerprinting is easy to recognize when analysing minified JavaScript,
as it works by enumerating various fonts to see if they are available on the fingerprinted
system. This can be done by writing text to a Canvas in various fonts and measuring
the rendered writing using the \textit{measureText} method \cite{englehardt2016census}.

Because of its usual use of the Canvas element, it is closely related to Canvas fingerprinting,
and it can be difficult to classify in some cases. In Section~\ref{sec:todataurl}
we have shown an example of a ``magic string'' pangram, which also contains some emoji.
It is used by \textit{Am I Unique?} and others to fingerprint via the Canvas element.
To that end, it is written onto a canvas in two different fonts, and some other elements are drawn
as well, and the pixel data is then extracted. This method obviously uses font information
to improve the fingerprint.
However, other methods are more font-centric and use many more fonts, as well as the \textit{measureText}
method.
Yet other methods do not use a Canvas element at all.

Font fingerprinting in the form where many fonts are enumerated is easy to spot in minified JavaScript,
as there is usually an array containing many font names present.

Within the context of Font fingerprinting, we have identified the ``magic string''
\textit{mmmmmmmmmmlli}. To the best of our knowledge, we are the first to notice this string
in relation to fingerprinting. It is used by at least ten sites in our dataset, and always occurs
with exactly ten times the letter ``m'', two times the letter ``l'', and one letter ``i''.

We are unsure about the real number of sites in our dataset that use this magic string, because
it is sometimes used within a \textit{span} element instead of a canvas.
JavaScript calls on this type of element are not instrumented by OpenWPM.
Thus, there may be more sites that use this magic string to fingerprint
in our dataset.
We have included an example of this type of fingerprinting in Figure~\ref{app:mmmmmmmmmmlli}
in the Appendix.

We speculate that the \textit{mmmmmmmmmmlli} string has certain subtle differences in width
when rendered by different browsers or font-rendering libraries on different systems, with
different fonts or font-families. We leave the determintation of the exakt reason to use
this particular string for future work.


\section{WebGL Fingerprinting}

\begin{figure}
\begin{minted}{sql}
SELECT DISTINCT site_url
FROM site_visits
  JOIN javascript
    ON site_visits.visit_id = javascript.visit_id
WHERE symbol LIKE '%getcontext%'
  AND arguments LIKE '%webgl%';
\end{minted}
\caption{Query to select all users of a WebGL context}
\label{code:webgl_query}
\end{figure}

\begin{table}
\centering
\caption{WebGL User's Behavior}
\begin{tabular}{l r r}
    \toprule
    & amount & percentage \\
    \midrule
    \textbf{Analyzed sites} & 75 & 100\% \\
    \midrule
    \textbf{Fingerprinting sites} & 34 & 45\% \\
    \midrule
    \textbf{Inconclusive} & 21 & 28\% \\
    too obfuscated & 10 & 13\% \\
    otherwise inconclusive & 11 & 15\% \\
    \midrule
    \textbf{Legitimate users} & 20 & 27\% \\
    \bottomrule
\end{tabular}
\label{table:webgl_users}
\end{table}

The results of our analysis, as displayed in Table~\ref{table:webgl_users}, again show that this
API is used by many fingerprinters in our dataset.
With WebGL, there is a higher number of sites which we have labeled ``inconclusive''.
This is because \todo{warum ist das so? war halt irgendwie schwerer zu erkennen...}

As OpenWPM in its current version does not instrument calls to the WebGL API, we have to
work around this by using the SQL query shown in Figure~\ref{code:webgl_query}.
There are no other symbols that OpenWPM instruments that are related to WebGL.

This also limits our methods of automatically labeling a script a fingerprinter.
The only information we can gather from the database OpenWPM creates
is whether a script creates a WenbGL context or not.

\todo{noch mehr zu WebGL? Was?}


\section{WebRTC Fingerprinting}

\begin{figure}
\begin{minted}{sql}
SELECT DISTINCT site_url
FROM site_visits
  JOIN javascript
    ON site_visits.visit_id = javascript.visit_id
WHERE symbol LIKE 'RTCPeerConnection%';
\end{minted}
\caption{Query to select all users of the WebRTC API}
\label{code:webrtc_query}
\end{figure}

\begin{table}
\centering
\caption{WebRTC User's Behavior}
\begin{tabular}{l r r}
    \toprule
    & amount & percentage \\
    \midrule
    \textbf{Analyzed sites} & 75 & 100\% \\
    \midrule
    \textbf{Fingerprinting sites} & 15 & 20\% \\
    \midrule
    \textbf{Inconclusive} & 51 & 68\% \\
    using adsafeprotected script & 37 & 49\% \\
    too obfuscated & 14 & 19\% \\
    otherwise inconclusive & 1 & 1\% \\
    \midrule
    \textbf{Legitimate users} & 9 & 12\% \\
    \bottomrule
\end{tabular}
\label{table:webrtc_users}
\end{table}


\section{Multiple Fingerprinting Techniques}
\label{sec:correlation}

\begin{table}
\centering
\caption{Numbers of Fingerprinting APIs Used}
\begin{tabular}{l r r}
    \toprule
    APIs used & amount & percentage \\
    \midrule
    \textbf{Analyzed sites} & 60 & 100\% \\
    \midrule
    1 & 15 & 25\% \\
    2 & 20 & 33\% \\
    3 & 13 & 22\% \\
    4 & 9 & 15\% \\
    5 & 2 & 3\% \\
    \bottomrule
\end{tabular}
\label{table:webrtc_users}
\end{table}

We assume that a site which was identified to fingerprint using one method is likely
to also be using another method.
This is a sane assumption, as more techniques give more entropy, and more entropy makes a more
unique and therefore better fingerprint.

To substantiate our claim, we have analysed 60 sites which we have previously identified as fingerprinters.
These sites were picked randomly from our set of already identified fingerprinters.

We exclude the basic attributes, such as the \textit{userAgent}, when we count the methods used by a site.
The counted methods are thus Audio-, Canvas-, Font-, WebGL- and WebRTC-Fingerprinting.
All analysed sites also used basic attributes to enhance their fingerprint.

The results are shown in Table~\todo{Tabelle}, and support our assumption. A quarter of the fingerprinters
use one API to fingerprint, and the remainder use more than one.
Between one and three APIs used are the most common cases, four and especially five different APIs are much less common.

We conclude that it is not especially rare for a site to use only one fingerprintable API in addition
to the basic browser attributes, but the majority of fingerprinters use more than one such API.


\section{Discussion}
In the following, we will discuss the acquired data and the analyses we have performed.

\subsection{Fingerprinters}
The data shows us that simply by using a Canvas object a certain way, or by using the Audio API,
a site is already a likely fingerprinter. We can assert this, because our analyses have shown
67\% of analysed users of \textit{toDataURL}, 72\% of analysed Audio API users and 45\%
of analysed users of WebGL to be fingerprinters.
Furthermore, 20\% of WebRTC users fingerprint, in addition to any fingerprinters we were unable
to identify, like the 49\% that use a suspicious but inconclusively labeled script.
Thus, just by using one or more of the listed APIs, we ascertain that a website is more likely
to be a fingerprinter than not.

We have also found certain giveaways of sites being fingerprinters.
The \textit{mmmmmmmmmmlli} magic string is used exclusively by font fingerprinting scripts.
The peculiar text \textit{Cwm fjordbank glyphs vext quiz} followed by an emoji is similarly used only
by fingerprinters.
Oftentimes, fingerprinter scripts contain very suspicious function names, such as \textit{getCanvasFingerprint}.
While the absence of these details does not help in classifying a site,
their presence can be used to instantly classify a site or script as a fingerprinter.


\subsection{Limitations}
When analysing a site, it is difficult or even impossible to tell whether an executed
script sends data back to the site's servers, or a third party.
To give an example, many sites include some manner of ``share'' button,
oftentimes for social media platforms like Facebook or Twitter.
These buttons are not always simply images surrounded by an HTML \textit{a}-tag,
but include JavaScript code as well.
This JavaScript, we assume, communicates with the network which ``owns''
the button, not the site where the button is displayed.
And this JavaScript may be a fingerprinter.

From a user's perspective, however, there may not be much of a difference.
In the end, the user is still getting fingerprinted and tracked, regardless
of who is at fault.


%%%%%%%%%%%%%%%%%%%%%%%%%%%%%%%%%%%%%%%%%%%%%%%%%%%%%%%%%%%%%%%%%%%%%%%%%%%%%%
%% Chapter End                                                              %%
%%%%%%%%%%%%%%%%%%%%%%%%%%%%%%%%%%%%%%%%%%%%%%%%%%%%%%%%%%%%%%%%%%%%%%%%%%%%%%


%%%%%%%%%%%%%%%%%%%%%%%%%%%%%%%%%%%%%%%%%%%%%%%%%%%%%%%%%%%%%%%%%%%%%%%%%%%%%%
%% Chapter Start                                                            %%
%%%%%%%%%%%%%%%%%%%%%%%%%%%%%%%%%%%%%%%%%%%%%%%%%%%%%%%%%%%%%%%%%%%%%%%%%%%%%%
\chapter{Design and Implementation}
In Chapter~\ref{chap:data_acquisition}, we have shown many users of certain JavaScript APIs to be fingerprinters.
We can build upon this knowledge to score websites in regards to their likelihood to be fingerprinting.

Our approach to implement automated fingerprinting detection is described in the following.
The software will analyse a website's behavior based on a data gathering by OpenWPM.
We will extract all information relevant to fingerprinting from the database,
and present this to the user of the detection software.
We will highlight any giveaways of fingerprinting, such as included magic strings
or suspicious function names.

In Section~\ref{sec:correlation}, we have shown that a fingerprinting site is more
likely to use a variety of fingerprinting methods than it is to use just one.
Thus, the more usage of suspicious APIs we detect, the higher the likelihood
that we have found a fingerprinter.

A site could be labeled more suspicious if many fingerprintable APIs are used within the same script,
and less suspicious if their use is spread across different scripts.
However, a site looking to avoid our detection mechanism could simply
spread their fingerprinting algorithm across multiple JavaScript files.
In addition to this, many modern websites use \textit{webpack}\footnote{\url{https://webpack.js.org/}}
or a similar technology to combine multiple JavaScript files into one when serving
them to their users. This could lead a non-fingerprinting site to be labeled
more likely to fingerprint than it actually is.


%%%%%%%%%%%%%%%%%%%%%%%%%%%%%%%%%%%%%%%%%%%%%%%%%%%%%%%%%%%%%%%%%%%%%%%%%%%%%%
%% Chapter End                                                              %%
%%%%%%%%%%%%%%%%%%%%%%%%%%%%%%%%%%%%%%%%%%%%%%%%%%%%%%%%%%%%%%%%%%%%%%%%%%%%%%

%%%%%%%%%%%%%%%%%%%%%%%%%%%%%%%%%%%%%%%%%%%%%%%%%%%%%%%%%%%%%%%%%%%%%%%%%%%%%%
%% Chapter Start                                                            %%
%%%%%%%%%%%%%%%%%%%%%%%%%%%%%%%%%%%%%%%%%%%%%%%%%%%%%%%%%%%%%%%%%%%%%%%%%%%%%%
\chapter{Conclusion and Outlook}
This thesis was terrific and there's lots of cool stuff to look out for.

%%%%%%%%%%%%%%%%%%%%%%%%%%%%%%%%%%%%%%%%%%%%%%%%%%%%%%%%%%%%%%%%%%%%%%%%%%%%%%
%% Chapter End                                                              %%
%%%%%%%%%%%%%%%%%%%%%%%%%%%%%%%%%%%%%%%%%%%%%%%%%%%%%%%%%%%%%%%%%%%%%%%%%%%%%%

% }}



% \begingroup
% \renewcommand{\cleardoublepage}{}
% \renewcommand{\clearpage}{}
% \chapter{Methods and Approach} % {{
% \endgroup
%
% \section{False Positives}
% Filtering false positives will prove to be the most challenging problem in this context.
% Requesting fonts via JavaScript might be a dead giveaway
% that fingerprinting code is, in fact, fingerprinting code. However, it might cause legitimate code, such
% as a multimedia player, to be mistaken for fingerprinting code, as well, even though it has a legitimate
% reason and perhaps a necessity to know a system's installed fonts.
% Meta-data can and possibly must be used to filter these; a list can be compiled of popular JavaScript libraries with legitimate,
% yet fingerprinting-like behavior, and then extended through analysis of collected data (see Technology below).
%
% All results must be evaluated regarding both precision and recall. Suppose a fingerprinting-recognition software
% analyzes 100 websites. It reports that 50 of them employ fingerprinting, and that the other 50 do not.
% In reality, however, only 40 of the 50 ``positives'' actually do employ fingerprinting, but a total
% of 80 of all the analyzed sites employ it.
% The precision of this software would then be $40/50 = 4/5$, as 40 out of 50 of its reported positives were correct.
% Its recall would be $40/80 = 1/2$, as it only reported 40 positives out of 80 actual positives.
% One might say that precision describes the usefulness of the results, while recall describes their completeness.
%
% \section{Metric}
% If a website includes a well-known browser fingerprinting library, or makes many calls to JavaScript functions that return
% data which is useful in building a fingerprint, one can conclude: this website is very likely generating,
% saving, and possibly distributing a browser's fingerprint; simply put: it performs browser fingerprinting.
% However, some modern websites may exhibit some similar traits or behavior that can be used to construct a fingerprint
% for reasons other than identification. Analyzing installed fonts has an obvious, legitimate use: correctly displaying
% text. Thus, it is important to work out a metric or rating system of some sort, which can
% (somewhat) reliably represent the likelihood that fingerprinting is, in fact, taking place.
% Acar et al. assert that the more fonts are being requested, the more likely a website is to
% be practicing fingerprinting \cite{DBLP:conf/ccs/AcarJNDGPP13}.
% This is based on findings of Eckersley, of the Panopticlick project \cite{eckersley2010unique}.
% Acar et al. further state that they classify a JavaScript file as a fingerprinter ``when it loads
% more than 30 fonts, enumerates plugins or mimeTypes, detects screen and navigator properties, and sends the
% collected data back to a remote server'' \cite{DBLP:conf/ccs/AcarJNDGPP13}.
% Englehardt and Narayanan present a metric to detect Canvas-based fingerprinting by specifying certain
% attributes such a Canvas, and the code using it, should have \cite{DBLP:conf/ccs/EnglehardtN16}.
% The above considerations form a good basis upon which to build a reliable metric, which will have to be
% evaluated further during the implementation process.
%
% \section{Scope}
% Finding or creating a single website which collects a fingerprint, and basing development of fingerprinting
% recognition on it, will not yield reliable or representative results. It is imperative to test
% the created recognition software against a multitude of fingerprinting libraries and users of these.
% It will be prudent to test the software against, say, the top 500 websites as ranked by
% Alexa\footnote{\url{http://www.alexa.com/topsites}}, as there is a high likelihood these sites employ
% a multitude of techniques to track their users, but there is no way to be sure if some of these sites wouldn't
% generate false positives.
% The safest way would be to create websites which employ an array of fingerprinting libraries both preexisting and
% implemented by the author, and then comparing the employed JavaScript calls with those from popular websites.
% The best approach seems to be a combination of the above; create a set of fake websites, some of which employ fingerprinting,
% others mimicking it. Then the results from analyzing them will be compared to the results from analyzing top Alexa sites.
%
% % }}

\begin{appendices}

\chapter{OpenWPM Browser Params}
The following configuration for OpenWPM's \textit{default\_browser\_params} was used.

\definecolor{mintedbg}{rgb}{0.95,0.95,0.95}
\setminted{fontsize=\footnotesize,bgcolor=mintedbg}

\begin{figure}
\label{app:params}
\begin{minted}{json}
{
    "extension_enabled": true,
    "cookie_instrument": true,
    "js_instrument": true,
    "cp_instrument": true,
    "http_instrument": true,
    "save_javascript": false,
    "save_all_content": false,

    "random_attributes": false,
    "bot_mitigation": false,
    "disable_flash": true,
    "profile_tar": null,
    "profile_archive_dir": null,
    "headless": true,
    "browser": "firefox",
    "prefs": {},

    "tp_cookies": "always",
    "donottrack": false,
    "disconnect": false,
    "ghostery": false,
    "https-everywhere": false,
    "adblock-plus": false,
    "ublock-origin": false,
    "tracking-protection": false
}
\end{minted}
\caption{The Browser Parameters used by OpenWPM in our Crawls}
\end{figure}


\chapter{\textit{mmmmmmmmmmlli} Magic String Use}
\begin{figure}
\begin{minted}[linenos]{javascript}
tb = function() {
    var a = ["monospace", "sans-serif", "serif"],
        b = document.getElementsByTagName("body")[0],
        c = document.createElement("div");
    c.setAttribute(
        "style",
        "visibility: hidden;position: absolute; top: 0px; left: -999px;"
    );
    b.appendChild(c);
    b = document.createElement("span");
    b.style.fontSize = "72px";
    b.innerHTML = "mmmmmmmmmmlli";
    var d = {},
        e = {},
        f;
    for (f in a) {
        b.style.fontFamily = a[f];
        c.appendChild(b);
        d[a[f]] = b.offsetWidth;
        e[a[f]] = b.offsetHeight;
        c.removeChild(b);
    }
    this.detect = function(b) {
        var f = document.createElement("div");
        f.setAttribute(
            "style",
            "visibility: hidden;position: absolute; top: 0px; left: -999px;"
        );
        for (var g = [], k = [], h = 0; h < b.length; h++) {
            var l = [];
            k.push(!1);
            for (var n in a) {
                var p = document.createElement("div"),
                    q = document.createElement("span");
                q.style.fontSize = "72px";
                q.innerHTML = "mmmmmmmmmmlli";
                q.style.fontFamily = b[h] + "," + a[n];
                p.appendChild(q);
                f.appendChild(p);
                l.push(q)
            }
            g.push(l)
        }
        c.appendChild(f);
        for (h = 0; h < b.length; h++)
            for (n in l = g[h], a) {
                q = l[n];
                p = q.offsetWidth != d[a[n]] || q.offsetHeight != e[a[n]];
                k[h] = k[h] || p;
            }
        c.removeChild(f);
        return k
    }
},
\end{minted}
\caption{Example of \textit{mmmmmmmmmmlli} magic string use by \url{bankofamerica.com}}
\label{app:mmmmmmmmmmlli}
\end{figure}

The code in Figure~\ref{app:mmmmmmmmmmlli} was taken from a script served by \url{bankofamerica.com}.
It is not a complete script, but rather just one function definition in a larger script.
We have ``beautified'' the code using DuckDuckGo's JavaScript beautifier, written by Akansh Gulati\footnote{\url{https://github.com/akanshgulati}}.
We have also modified the beautified code further to eliminate long lines, thus allowing it to be displayed
on this page.
We have not made any functional modifications.

The code demonstrates the use of the \textit{mmmmmmmmmmlli} magic string in a \textit{span} HTML element.
It is used for font fingerprinting without the use of a Canvas element.
The original URL of the full JavaScript was \url{https://secure.bankofamerica.com/login/sign-in/cc.go}.

In the code, multiple \textit{span} elements are created and inserted into the DOM,
each containing the magic string with a font size of 72 pixels, and with their font-family
set to different combinations of fonts and the font types ``monospace'', ``sans-serif'' and ``serif''.
These elements are invisible from the user.
Their \textit{offsetWidth} and \textit{offsetHeight} properties are read, saved, and presumably
used for fingerprinting.
The list of fonts used contains 300 entries, but is not shown for brevity's sake.


\end{appendices}

\clearpage

\printbibliography

\end{document}
