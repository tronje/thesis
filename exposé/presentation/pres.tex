%----------------------------------------------------------------------------------------
%	PACKAGES AND THEMES
%----------------------------------------------------------------------------------------

\documentclass{beamer}

\mode<presentation> {

\usetheme{Dresden}


% Colors

\usecolortheme{beaver}

\setbeamertemplate{navigation symbols}{} % To remove the navigation symbols from the bottom of all slides uncomment this line
}

% \usepackage[german,ngerman]{babel}
\usepackage[utf8]{inputenc}
\usepackage[T1]{fontenc}
\usepackage{lmodern}
\usepackage{amssymb}
\usepackage{mathtools}
\usepackage{amsmath}

\usepackage{graphicx} % Allows including images
\usepackage{booktabs} % Allows the use of \toprule, \midrule and \bottomrule in tables

%----------------------------------------------------------------------------------------
%	TITLE PAGE
%----------------------------------------------------------------------------------------

\title[BSc Thesis]{Browser Fingerprinting} 
\author[Tronje Krabbe]{Tronje Krabbe\newline\url{tronje@informatik.uni-hamburg.de}}
\institute[UHH]
{
Uni Hamburg \\
Working Group on Security and Privacy\\
}
\date{\today}

\begin{document}

\begin{frame}
\titlepage
\end{frame}

\begin{frame}
\frametitle{Overview}
\tableofcontents
\end{frame}

%\AtBeginSection{
%\begin{frame}
%\frametitle{Overview} 
%\tableofcontents[currentsection]
%\end{frame}}

%----------------------------------------------------------------------------------------
%	PRESENTATION SLIDES
%----------------------------------------------------------------------------------------

%------------------------------------------------
\section{Browser Fingerprinting}
%------------------------------------------------

\subsection{}

%------------------------------------------------

\begin{frame}
    \frametitle{What is that?}
    The average browser exposes enough information to construct
    a `fingerprint', which can uniquely identify a user. Information used can be:
    \begin{itemize}
        \item user agent
        \item installed fonts
        \item GPU model/vendor
        \item WebGL behavior
        \item ...
    \end{itemize}
\end{frame}

%------------------------------------------------

\begin{frame}
    \frametitle{Why is it `bad'?}
    \begin{itemize}
        \item users are usually not made aware
        \item more easily obfuscated than a cookie
        \item cannot be easily avoided
    \end{itemize}

    \begin{figure}[h]
    \centering
    \includegraphics[width=\textwidth]{cookies.png}
    \caption{Reddit.com's cookie warning}
    \end{figure}

\end{frame}

%------------------------------------------------

\section{Thesis}
\subsection{}

%------------------------------------------------

\begin{frame}
    \frametitle{Thesis}
    Develop a software that can reliably detect whether a website employs fingerprinting,
    and integrate it into PrivacyScore.org
\end{frame}

%------------------------------------------------

\begin{frame}
    \frametitle{How?}
    Simply record all JavaScript function calls, and see if it looks like a fingerprint is being
    constructed.
\end{frame}

%------------------------------------------------

\section{Considerations}
\subsection{}

%------------------------------------------------

\begin{frame}
    \frametitle{Identify most commonly used fingerprinting methodologies}
    \begin{itemize}
        \item Flash can be used, but unlikely to be widespread
        \item fonts and WebGL are popular, what about e.g. sound?
        \item how likely will these continue to be used?
        \item JavaScript and WebGL best candidates
    \end{itemize}
\end{frame}

%------------------------------------------------

\begin{frame}
    \frametitle{Identify most commonly used libraries}
    \begin{itemize}
        \item site that uses a common library is easily `convicted'
        \item could be mostly in-house libs $\rightarrow$ pretty much useless
        \item JavaScript is easily obfuscated
    \end{itemize}
    
\end{frame}

%------------------------------------------------

\begin{frame}
    \frametitle{False Positives}
    JavaScript calls that look like they're constructing a fingerprint may also be
    used for a completely different cause. Recall:
    \begin{itemize}
        \item user agent
        \item installed fonts
        \item GPU model/vendor
        \item WebGL behavior
        \item ...
    \end{itemize}
\end{frame}

%------------------------------------------------

\begin{frame}
    \frametitle{Scoring System}
    Instead of reporting \textbf{``yes, fingerprinting is going on!''} \\
    or \textbf{``no, fingerprinting is not going on!''}, give a score
    to represent a likelihood.
\end{frame}

%------------------------------------------------

\begin{frame}
    \frametitle{Performance}
    \begin{itemize}
        \item efficient algorithm
        \item language choice
        \begin{itemize}
            \item Python
            \begin{itemize}
                \item slow
                \item OpenWPM \& PrivacyScore are written in Python
            \end{itemize}
            \item C++, Rust
            \begin{itemize}
                \item hard to integrate
                \item more complicated than Python
            \end{itemize}
        \end{itemize}
    \end{itemize}
\end{frame}

%------------------------------------------------

\section{Schedule}
\subsection{}

%------------------------------------------------

\begin{frame}
    \frametitle{Schedule}
    \scriptsize
    \textit{``The first 90 percent of the code accounts for the first 90 percent of the development time. The remaining 10 percent of the code accounts for the other 90 percent of the development time.''} - Tom Cargill, Bell Labs
    \\
    \normalsize
    \begin{itemize}
        \item 4 weeks: first implementation and tests in place
        \item 6 weeks: evaluation and improvements
        \item 4 weeks: writing the actual thesis paper
        \item about 1 month to spare
    \end{itemize}
\end{frame}

%------------------------------------------------

\section{}
\begin{frame}[c]
    \begin{center}
    \Huge Thank you!
    \end{center}
\end{frame}

%------------------------------------------------

%----------------------------------------------------------------------------------------

\end{document}
