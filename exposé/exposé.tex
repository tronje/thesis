\documentclass[a4paper, 12pt]{scrreprt}

\usepackage[utf8]{inputenc}
% \usepackage{datetime2}
\usepackage{amsmath}
\usepackage{marvosym}
\usepackage{graphicx}
\usepackage{parskip}
\usepackage{hyperref}
\usepackage{setspace}
\usepackage{acronym}
\usepackage{float}

\usepackage[
    url=true,
    style=numeric,
    alldates=long,
    dateabbrev=false
]{biblatex}
\addbibresource{../literature.bib}

\titlehead{
    \begin{center}
        % \includegraphics[width=4cm]{../images/uhh_logo_minimal.png}\\
        \includegraphics[width=8cm]{../images/UHH-Logo_2010_Farbe_RGB_hires_nomargin.png}\\
        \medskip
        Department of Computer Science
    \end{center}
}

\title{
    Exposé\\
    \medskip
    \large Implementation of browser fingerprinting recognition \\
    and integration into PrivacyScore
}

\author{Tronje Krabbe}

\date{\today}

\begin{document}
\hypersetup{hidelinks}

\maketitle

\begin{abstract}
    \doublespacing
    This is an overview of the author's thesis. It describes the user identification and tracking technique `browser fingerprinting',
    and explores the possibilities and technicalities in regards to detecting wether a website employs this technique.
\end{abstract}

\tableofcontents

\chapter{Introduction}
When browsing the web, users can be tracked through the use of cookies, which store information on the user's computer.
If the user wishes not to be trackable, they can modify or delete any cookie as they see fit. This interaction gives
operators of websites the ability to track users only if they wish to be tracked.
There are, however, other methods of uniquely identifying a user and tracking them during their use of the internet,
which is not as easily mitigated as a cookie.\cite{am_i_unique} On of these methods is called `browser fingerprinting', and it will be
the focus of this thesis.

Tracking a user does not simply mean recognizing whether a visitor to one's website is a new or an existing user.
Identifying information can be passed on or sold to partners, advertisers, or even governments, in order to
construct a rich browsing history of a user.

Browser fingerprinting, also known as device fingerprinting, works by analyzing a web browser's configuration and settings,
such as installed fonts, language settings, time zone settings, installed add-ons, to name a few.
These attributes are readliy available to be collected through JavaScript functions. Simply recording
all function calls made by the JavaScript frontend of a website can reveal whether fingerprinting is likely to
be taking place or not.\cite{faiz2014browser}\cite{panopticlick}

    \subsection{Leading Question and Goals}
    The thesis presents the assertion that techniques to identify and track users across different websites without
    their knowledge or their ability to easily intervene, violates their privacy.
    The author will therefore attempt to implement a technique to recognize when a website is deploying browser fingerprinting,
    and along the way explore the techniques used to do so.
    The ultimate goal is to integrate the implementation into \url{https://privacyscore.org}, a web-service to test and rank
    websites according to the extent to which they respect their users' privacy.

    \subsection{Methods and Approach}
    \begin{description}
        \item[Metric] \hfill \\
            If a website includes a browser fingerprinting library, or makes many calls to JavaScript functions that return
            data which is useful in building a fingerprint, one can conclude: this website is very likely to actually
            generate, save, and possibly distribute a browser's fingerprint; simply put: it does browser fingerprinting.
            However, most modern websites use at least some of the attributes that can be used to construct a fingerprint
            for reasons other than identification. Analysing installed fonts has an obvious, legitimate use: correctly displaying
            fonts. Thus, it is important to work out a metric or rating system of some sort, which can
            (somewhat) reliably represent the likelyhood that fingerprinting is, in fact, taking place.

        \item[Scope] \hfill \\
            Finding or creating a single website which collects a fingerprint, and basing development of fingerprinting
            recognition on it, will not yield reliable or representative results. It is imperative to test
            the created recognition software against a multitude of fingerprinting libraries and users of these.

            TODO: how?
    \end{description}

\chapter{Implementation}
TODO

    \subsection{Technology}
    TODO

    \begin{description}
        \item[OpenWPM] \hfill \\
            ``OpenWPM is a web privacy measurement framework which makes it easy to collect data for privacy studies on a scale
            of thousands to millions of site''\footnote{\url{https://github.com/citp/OpenWPM}}.
            The author is planning to build upon OpenWPM\cite{englehardt2016census} to create a software that can detect
            browser fingerprinting. OpenWPM includes capabilities to record
            ``all method calls (with arguments) and property accesses for APIs of potential fingerprinting interest'',
            providing a good basis for data collection. Analysing and interpreting this data in a meaningful way
            will need to be implemented by the author.
    \end{description}

\clearpage

\printbibliography

\end{document}
